\documentclass{article}
\begin{document}
\begin{enumerate}
\item \texttt{AttributeClassExtractor}\\ 
Kelas ini merupakan kelas untuk mengambil informasi sebuah atribut yang terdapat pada kelas.

Kelas ini tidak memiliki atribut. \textit{Method-method} yang dimiliki kelas ini adalah sebagai berikut.
\begin{itemize}
\item \texttt{public static void extractAttributeClassContent(FieldDoc[] fields, BufferedWriter out)}\\ 
\textit{Method} ini akan menampilkan atribut-atribut yang dimiliki oleh sebuah kelas

\textbf{Parameter:}
\begin{itemize}
\item \texttt{FieldDoc[] fields} - 
sebuah array berisikan sejumlah atribut dari kelas
\item \texttt{BufferedWriter out} - 
turunan dari kelas \texttt{Writer} yang digunakan untuk menulis file text
\end{itemize}
\textbf{Kembalian}: Tidak memiliki \textit{return value}

\textbf{Exception}: IOException

\end{itemize}
\item \texttt{ClassExtractor}\\ 
Kelas ini merupakan kelas untuk mengambil informasi dari sebuah kelas.

Kelas ini tidak memiliki atribut. \textit{Method-method} yang dimiliki kelas ini adalah sebagai berikut.
\begin{itemize}
\item \texttt{public static void extractClassContent(ClassDoc[] classes, BufferedWriter out)}\\ 
\textit{Method} ini akan menampilkan nama kelas berserta penjelasan dari sebuah kelas

\textbf{Parameter:}
\begin{itemize}
\item \texttt{ClassDoc[] classes} - 
sebuah array berisikan sejumlah kelas
\item \texttt{BufferedWriter out} - 
turunan dari kelas \texttt{Writer} yang digunakan untuk menulis file text
\end{itemize}
\textbf{Kembalian}: Tidak memiliki \textit{return value}

\textbf{Exception}: IOException

\end{itemize}
\item \texttt{Extractor}\\ 
Kelas ini merupakan kelas untuk menjalan \textit{custom doclet}.

Atribut yang dimiliki kelas ini adalah sebagai berikut.
\begin{itemize}
\item \texttt{String fileName} - atribut untuk nama \textit{file}
\end{itemize}
\textit{Method-method} yang dimiliki kelas ini adalah sebagai berikut.
\begin{itemize}
\item \texttt{public static boolean start(RootDoc root)}\\ 
\textit{Method} ini berperan sebagai \textit{method} untuk menjalankan
 \textit{custom doclet}

\textbf{Parameter:}
\begin{itemize}
\item \texttt{RootDoc root} - 
berperan sebagai mengambil seluruh informasi spesifik dari
 \textit{option} yang terdapat pada \textit{command-line} sebuah
 \textit{terminal}. Selain itu berperan juga untuk mengambil informasi dari
 sekumpulan \textit{file java} yang akan di proses.
\end{itemize}
\textbf{Kembalian}: kondisi true

\textbf{Exception}: Tidak memiliki \textit{exception}

\item \texttt{public static void init(ClassDoc[] classes)}\\ 
\textit{Method} ini berperan untuk menulis kedalam sebuah \textit{file}
 saat \textit{javadoc} berjalan.

\textbf{Parameter:}
\begin{itemize}
\item \texttt{ClassDoc[] classes} - 
sebuah array yang berisikan sekumpulan \textit{file java}
 yang akan di proses.
\end{itemize}
\textbf{Kembalian}: Tidak memiliki \textit{return value}

\textbf{Exception}: IOException

\item \texttt{public static int optionLength(String option)}\\ 
Method untuk menghitung banyak option yang digunakan pada \textit{command-line}

\textbf{Parameter:}
\begin{itemize}
\item \texttt{String option} - 
sebuah option
\end{itemize}
\textbf{Kembalian}: panjang setiap option

\textbf{Exception}: Tidak memiliki \textit{exception}

\item \texttt{public static boolean validOptions(java.lang.String[][] args, DocErrorReporter err)}\\ 
Pengecekan option valid

\textbf{Parameter:}
\begin{itemize}
\item \texttt{java.lang.String[][] args} - 
String array 2 dimensi dari option
\item \texttt{DocErrorReporter err} - 
sebuah error jika tidak terdapat option tersebut.
\end{itemize}
\textbf{Kembalian}: bernilai true jika option tersebut dikenali, false jika option tersebut tidak dikenali

\textbf{Exception}: Tidak memiliki \textit{exception}

\end{itemize}
\item \texttt{MethodClassExtractor}\\ 
Kelas ini merupakan kelas untuk mengambil informasi sebuah \textit{method}
 terdapat pada kelas.

Kelas ini tidak memiliki atribut. \textit{Method-method} yang dimiliki kelas ini adalah sebagai berikut.
\begin{itemize}
\item \texttt{public static void extractMethodContent(ClassDoc superclass, MethodDoc[] methods, BufferedWriter out)}\\ 
\textit{Method} ini akan menampilkan \textit{method-method} yang dimiliki oleh
 sebuah kelas

\textbf{Parameter:}
\begin{itemize}
\item \texttt{ClassDoc superclass} - 
sebuah objek ClassDoc
\item \texttt{MethodDoc[] methods} - 
sebuah array berisikan sejumlah \textit{method} dari kelas
\item \texttt{BufferedWriter out} - 
turunan dari kelas \texttt{Writer} yang digunakan untuk menulis
 file text
\end{itemize}
\textbf{Kembalian}: Tidak memiliki \textit{return value}

\textbf{Exception}: IOException

\end{itemize}
\end{enumerate}
\end{document}
