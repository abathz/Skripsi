\chapter{Kesimpulan dan Saran}
\label{sec:kesimpulan dan saran}

\section{Kesimpulan}
\label{sec:kesimpulan}
Dari hasil pembangunan perangkat lunak Konversi Javadoc ke \LaTeX\ , didapatkanlah kesimpulan-kesimpulan sebagai berikut:
\begin{itemize}
	\item Telah berhasil mengimplementasi {\it Javadoc Doclet} API untuk mengambil informasi dari sekumpulan {\it file Java} yang terdapat pada sebuah {\it package} dan mengonversikannya dari format {\it Javadoc} ke format \LaTeX.
\end{itemize}

\section{Saran}
\label{sec:saran}
Dari hasil penelitian termasuk kesimpulan yang didapat, berikut adalah beberapa saran untuk pengembangan.
\begin{enumerate}
	\item Pada saat ini, perangkat lunak tidak mengatasi {\it tag-tag html}. Pada pengujian menggunakan SIAModels terdapat beberapa kelas yang memiliki {\it tag-tag} html pada {\it javadoc} seperti contoh pada kelas {\tt Mahasiswa} pada fungsi {\tt calculateIPKLulus()}. Terdapat kode yang tidak lazim yaitu tanda \textexclamdown\ yang merepresentasikan tanda lebih kecil "<" dan tanda \textquestiondown\ yang merepresentasikan tanda lebih besar ">". Sebaiknya perangkat lunak dapat mengatasi {\it tag-tag html} jika javadoc memiliki {\it tag-tag} tersebut.
	\item Pada saat ini, perangkat lunak menampilkan tulisan "tidak memiliki {\it exception}" pada bagian {\bf Exception}. Sebaiknya mengikuti standar dari {\it javadoc} bahwa jika {\it method} yang tidak memiliki {\it exception} maka bagian {\bf Exception} tidak akan muncul pada dokumentasi.
\end{enumerate}
