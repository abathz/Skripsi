\chapter{Analisis}
\label{sec:analisis}

Bab ini membahas mengenai analisis struktur \LaTeX\ .

\section{Analisis Struktur \LaTeX}
\label{sec:struktur}
Struktur \LaTeX\ yang akan dibahas pada skripsi ini memiliki format sebagai berikut.
\begin{enumerate}
	\item {\it List level} pertama\\
	Pada {\it list level} pertama ini akan menampilkan sebuah nama {\it class} dan penjelasan terkait dengan {\it class} ini. {\it List} yang dibuat akan menggunakan {\it ordered list} dengan {\it command} \texttt{\string\begin\{enumerate\}}\dots\texttt{\string\end\{enumerate\}} dan {\it command} \texttt{\string\texttt\{namaKelas\}} akan digunakan untuk menampilkan nama {\it class}.
	\item {\it List level} kedua\\
	Pada {\it list level} kedua ini terdapat dua {\it list} yang masing-masing akan menampilkan atribut-atribut dan {\it method-method} yang dimiliki oleh {\it class} ini. {\it List} pertama yang dibuat akan menggunakan {\it unordered list} dengan {\it command} \texttt{\string\begin\{itemize\}}\dots\texttt{\string\end\{itemize\}} untuk mengisi atribut-atribut yang terdapat pada {\it class} ini. {\it Command} \texttt{\string\texttt\{atribut\}} akan digunakan untuk menampilkan atribut. Atribut ini menampilkan tipe atribut dan nama atribut. {\it List} yang kedua akan menggunakan {\it unordered list} dengan {\it command} \texttt{\string\begin\{itemize\}}\dots\texttt{\string\end\{itemize\}} untuk mengisi {\it method-method} yang terdapat pada {\it class} ini dan penjelasan terkait dengan {\it method} tersebut. {\it Command} \texttt{\string\texttt\{method\}} akan digunakan untuk menampilkan {\it method}. {\it Method} ini menampilkan {\it access modifier} dari {\it method}, tipe kembalian {\it method} dan nama {\it method}.
	\item {\it List level} ketiga\\
	Pada {\it list level} ketiga ini akan menampilkan parameter yang digunakan pada {\it method}. {\it List} yang dibuat akan menggunakan {\it unordered list} dengan {\it command} \texttt{\string\begin\{itemize\}}\dots\texttt{\string\end\{itemize\}} dan {\it command} \texttt{\string\texttt\{parameter\}} akan digunakan untuk menampilkan parameter. Parameter ini menampilkan nama parameter. 
\end{enumerate}