%\lstdefinestyle{mystyle}{
%    basicstyle=\footnotesize,                
%   	captionpos=b,                     
%   	numbers=left,                    
%   	numbersep=5pt,             
%   	tabsize=2,
%   	breaklines
%}
%\lstset{style=mystyle}
\chapter{Implementasi dan Pengujian}
\label{sec: Implementasi dan Pengujian}

Bab ini terdiri atas dua bagian, yaitu Implementasi Perangkat Lunak dan Pengujian Perangkat Lunak. Bagian implementasi berisi perintah menggunakan perangkat lunak dan langkah-langkah dalam pengunaan perangkat lunak. Sedangkan bagian pengujian berisi hasil pengujian fungsional dan pengujian eksperimental.
\section{Implementasi Perangkat Lunak}
\label{sec: Implementasi Perangkat Lunak}

Perangkat lunak dibuat menggunakan bahasa {\it java} dan dimasukkan ke dalam sebuah {\it jar} sehingga dapat digunakan dengan cara mengeksekusi perintah. Penggunaan {\it file jar} tersebut dapat dilakukan melalui {\it Terminal} di Linux/Mac atau {\it Command Prompt} di Windows. Berikut perintah yang digunakan untuk menjalankan perangkat lunak.
\begin{verbatim}
	javadoc -filename <file-name>
	        -classpath <pathlist>
	        -doclet extractor.Extractor
	        -docletpath GenerateJavadocToLatex.jar
	        -sourcepath <pathlist>
	        -subpackages <subpkglist>
	        <packagenames>
\end{verbatim}
Potongan perintah di atas memiliki 2 parameter yaitu {\it option} dan {\it packagenames}. Parameter {\it packagename} adalah parameter untuk {\it package} yang akan digunakan sebagai masukan dari perangkat lunak. Parameter {\it option} adalah beberapa perintah pendukung. Berikut beberapa perintah {\it option} yang digunakan untuk mendukung berjalannya perangkat lunak.
\begin{itemize}
	\item {\tt -filename <file-name>} - Menghasilkan {\it output file} dengan nama {\tt file-name.tex}. Jika {\it option} ini tidak digunakan maka nama {\it file} yang dihasilkan akan bernama {\tt doc.tex}.
	\item {\tt -classpath <pathlist>} - Letak {\it file} {\tt jar} yang dipergunakan. Perintah ini bersifat opsional jika kelas tersebut membutuhkan kelas yang tidak ada di dalam standar {\it library Java}.
	\item {\tt -doclet <class>} - Kelas yang dibuat untuk menghasilkan {\it output}. Pada kasus ini adalah {\tt extractor.Extractor}.
	\item {\tt -docletpath <path>} - Letak {\it doclet} yang sudah di-{\it package} menjadi {\it file jar}.
	\item {\tt -sourcepath <pathlist>} - Letak {\it source file} sebagai masukan. 
	\item {\tt -subpackages <subpkglist>} - Letak {\it subpackage} yang akan dimuat secara rekursif. Perintah ini dapat digunakan secara opsional jika terdapat sebuah {\it subpackage} di dalam {\it package} yang menjadi masukan.
	\item {\tt <packagenames>} - Nama {\it package} yang menjadi masukan. Lokasi {\it package} bergantung pada {\tt -sourcepath} yang dituju.
\end{itemize}

Untuk penggunaan perintah di atas, Langkah pertama membuka aplikasi {\it Terminal} atau {\it Command Prompt}. Langkah kedua mengetik perintah {\tt javadoc} lalu diikuti dengan perintah pendukungnya seperti yang sudah dijelaskan di atas. Berikut contoh perintah lengkap yang digunakan.

\begin{verbatim}
	javadoc -filename siamodels
	        -doclet extractor.Extractor
	        -docletpath GenerateJavadocToLatex.jar
	        -sourcepath SIAModels/src/main/java
	        -subpackages id.ac.unpar.siamodels
	        id.ac.unpar.siamodels
\end{verbatim}

\subsection{Lingkungan Implementasi}
\label{sec:lingkungan perangkat lunak}
Dalam proses pengujian perangkat lunak ini digunakan spesifikasi perangkat sebagai berikut.

\begin{enumerate}
	\item Processor: Intel Core i7 2.5-3.7GHz 
	\item RAM: 16.00 GB DDR3	
	\item Harddisk : 512MB SSD
	\item VGA : Intel Iris Pro dan AMD Radeon R9 M370X
	\item Sistem Operasi: macOS High Sierra
	\item Versi Java: 1.8.0\_121
	\item Code Editor: Netbeans 8.2
\end{enumerate}

\section{Pengujian Perangkat Lunak}
\label{sec: pengujian perangkat lunak}
Pada sub bab ini akan menjelaskan Pengujian Fungsional dan Pengujian Eksperimental. Pengujian Fungsional akan menguji perangkat lunak terhadap kode program sederhana serta menguji kode program perangkat lunak yang dibuat. Pengujian Eksperimental akan menguji perangkat lunak terhadap kode program SIAModels.

\subsection{Pengujian Fungsional}
\label{sec:pengujian fungsional}
Pada pengujian fungsional dilakukan terhadap 2 kasus yaitu pengujian kode program sederhana dan kode program perangkat lunak yang dibuat.

Pada kasus pertama dijalankan perintah sebagai berikut.

\begin{verbatim}
	javadoc -filename operasimatematika
	        -doclet extractor.Extractor
	        -docletpath GenerateJavadocToLatex.jar
	        -sourcepath OperasiMatematika/src
	        operasi
\end{verbatim}

Perintah tersebut akan mengambil informasi dari kode program yang dapat dilihat pada Lampiran \ref{kodeProgram:A}. Setelah perintah tersebut dieksekusi akan menghasilkan {\it file} \LaTeX\ yang dapat dilihat pada Lampiran \ref{hasilLatex:A}. Jika {\it file} \LaTeX\ tersebut diekseskusi menggunakan perintah {\tt pdflatex} akan menghasil sebuah {\it file} PDF yang dapat dilihat pada Lampiran \ref{hasilPDF:A}.

Dapat dilihat dari hasil yang terdapat Lampiran \ref{hasilPDF:A}, didapatkan 5 kelas yang dikonversi menjadi 5 buah {\it item list enumarate} yang masing berisikan nama kelas dari 5 file tersebut. {\it Method-method} yang terdapat pada kelas tersebut akan dibuat dengan menggunakan {\it list itemize} sebanyak {\tt n} buah {\it method} dan parameter dari {\it method} tersebut dibuat menggunakan {\it list itemize} sebanyak {\tt n} buah parameter.

Pada kasus kedua dijalankan perintah sebagai berikut.

\begin{verbatim}
	javadoc -filename javadoctolatex
	        -classpath tools.jar
	        -doclet extractor.Extractor
	        -docletpath GenerateJavadocToLatex.jar
	        -sourcepath GenerateJavadocToLatex/src
	        extractor
\end{verbatim}

Perintah tersebut akan mengambil informasi dari kode program yang dapat dilihat pada Lampiran \ref{kodeProgram:B}. Setelah perintah tersebut dieksekusi akan menghasilkan {\it file} \LaTeX\ yang dapat dilihat pada Lampiran \ref{hasilLatex:B}. Jika {\it file} \LaTeX\ tersebut diekseskusi menggunakan perintah {\tt pdflatex} akan menghasilkan sebuah {\it file} PDF yang dapat dilihat pada Lampiran \ref{hasilPDF:B}.

Dapat dilihat dari hasil yang terdapat Lampiran \ref{hasilPDF:B}, didapatkan 4 kelas yang dikonversi menjadi 4 buah {\it item list enumarate} yang masing berisikan nama kelas dari 4 file tersebut. Atribut kelas akan dibuat dengan menggunakan {\it list itemize} sebanyak {\tt n} buah atribut. {\it Method-method} yang terdapat kelas tersebut akan dibuat dengan menggunakan {\it list itemize} sebanyak {\tt n} buah {\it method} dan parameter dari {\it method} tersebut dibuat menggunakan {\it list itemize} sebanyak {\tt n} buah parameter.

Pada kasus ketiga dijalankan perintah sebagai berikut.

\begin{verbatim}
	javadoc -classpath tools.jar
	        -doclet extractor.Extractor
	        -docletpath GenerateJavadocToLatex.jar
	        GenerateJavadocToLatex/src/MyApp.java
\end{verbatim}

Perintah tersebut akan mengambil informasi dari kode program yang dapat dilihat pada Lampiran \ref{kodeProgram:C}. Setelah perintah tersebut dieksekusi akan menghasilkan {\it file} \LaTeX\ yang dapat dilihat pada Lampiran \ref{hasilLatex:C}. Jika {\it file} \LaTeX\ tersebut diekseskusi menggunakan perintah {\tt pdflatex} akan menghasilkan sebuah {\it file} PDF yang dapat dilihat pada Lampiran \ref{hasilPDF:C}.

Dapat dilihat dari hasil yang terdapat pada Lampiran \ref{hasilPDF:C}, didapatkan kurangnya penjelasan deskripsi kelas, deskripsi atribut, deskripsi {\it method} dan penjelasan {return value}.

\subsection{Pengujian Eksperimental}
\label{sec:pengujian eksperimental}
Pengujian eksperimental dilakukan terhadap kode program SIAModels. SIAModels adalah sekumpulan kelas {\it java} yang mewakili objek-objek Sistem Informasi Akademik Universitas Katolik Parahyangan. SIAModels ini berisikan objek matakuliah pada Fakultas Teknologi Informasi dan Sains.

Pada pengujian menggunakan SIAModels dilakukan perintah sebagai berikut.

\begin{verbatim}
	javadoc -filename siamodels
	        -doclet extractor.Extractor
	        -docletpath GenerateJavadocToLatex.jar
	        -sourcepath SIAModels/src/main/java
	        -subpackages id.ac.unpar.siamodels
	        id.ac.unpar.siamodels
\end{verbatim}

Perintah tersebut akan mengambil informasi dari kode program yang dapat dilihat pada Lampiran \ref{kodeProgram:D}. Setelah perintah tersebut dieksekusi akan menghasilkan {\it file} \LaTeX\ yang dapat dilihat pada Lampiran \ref{hasilLatex:D}. Jika {\it file} \LaTeX\ tersebut diekseskusi menggunakan perintah {\tt pdflatex} akan menghasilkan sebuah {\it file} PDF yang dapat dilihat pada Lampiran \ref{hasilPDF:D}.

Dapat dilihat dari hasil yang terdapat Lampiran \ref{hasilPDF:D}, didapatkan {\tt n} buah kelas dan {\tt n} buah {\it innerclass} yang dikonversi menjadi {\tt n} buah {\it item list enumarate} yang masing-masing berisikan nama kelas dan {\it innerclass} dari {\tt n} {\it file} tersebut. Atribut kelas akan dibuat dengan menggunakan {\it list itemize} sebanyak {\tt n} buah atribut. {\it Method-method} yang terdapat kelas tersebut akan dibuat dengan menggunakan {\it list itemize} sebanyak {\tt n} buah {\it method} dan parameter dari {\it method} tersebut dibuat menggunakan {\it list itemize} sebanyak {\tt n} buah parameter. Terdapat beberapa ketidaklaziman yang ditemukan setelah mengeksekusi perintah {\tt pdflatex}.

\begin{itemize}
	\item Terdapat karakter yang tidak lazim yaitu tanda \textexclamdown\ yang merepresentasikan tanda lebih kecil "<" dan tanda \textquestiondown\ yang merepresentasikan tanda lebih besar ">". Karakter tersebut muncul pada kelas: 
		\begin{enumerate}
			\item {\tt TahunSemester} - pada atribut kelas
			\item {\tt Mahasiswa} - pada {\it method} {\tt calculateIPKLulus()}, {\tt calculateIPLulus()}, {\tt calculateIPTempuh()}, {\tt calculateIPKumulatif()}, {\tt calculateIPKTempuh()} dan {\tt calculateIPS()}
		\end{enumerate}
		Karakter yang tidak lazim tersebut muncul karena secara otomatis \LaTeX\ menggunakan OT1 {\it font encoding}. Pada {\it Ascii Table} tanda \textless\ memiliki kode {\tt 3C} dan tanda \textgreater\ memiliki kode {\tt 3E} dan secara otomatis \LaTeX\ akan memetakan tanda \textless\ dan tanda \textgreater\ menjadi tanda \textexclamdown\ dan \textquestiondown\ ~\cite{ot1fontencoding:04:ot1fontencoding}. Untuk dapat memunculkan tanda \textless\ dan tanda \textgreater\ terdapat 2 cara yaitu dengan menambahkan {\it package} {\tt \string\usepackage[T1]\{fontenc\}} atau menggunakan command {\tt \string\textless} sebagai \textless\ dan {\tt \string\textgreater} sebagai \textgreater.
	\item Beberapa deskripsi tidak muncul pada hasil PDF, contohnya seperti pada kelas:
		\begin{enumerate}
			\item {\tt Semester} - pada deskripsi kelas, atribut kelas dan {\it method}
			\item {\tt TahunSemester} - pada deskripsi {\it method} seperti pada {\it method} {\tt getTahun()}
			\item {\tt Mahasiswa} - pada deskripsi kelas, atribut kelas dan beberapa {\it method}
			\item {\tt Mahasiswa.Nilai} - pada beberapa deskripsi {\it method}
			\item {\tt JadwalKuliah} - pada deskrpisi kelas, atribut kelas dan {\it method}
			\item {\tt MataKuliah} - pada deskripsi kelas, atribut kelas, dan {\it method}
			\item {\tt Dosen} - pada deskripsi kelas, atribut kelas dan {\it method}
		\end{enumerate}
		Setelah dilakukan pemeriksaan pada hal tersebut, kesalahan yang terjadi bukan dari perangkat lunak melainkan kode program yang menjadi masukan memiliki dokumentasi yang kurang lengkap.
	\item Parameter setiap {\it method} memiliki tipe variabel yang lengkap. Setelah dilakukan pemeriksaan, tipe variabel yang bersifat {\it Object} akan menghasilkan tipe variabel yang {\it fully-qualified} seperti contohnya pada tipe variabel {\tt String} yang akan menghasilkan {\tt java.lang.String}.
\end{itemize}






















