\lstdefinestyle{mystyle}{
    basicstyle=\footnotesize,                
   	captionpos=b,                     
   	numbers=left,                    
   	numbersep=5pt,             
   	tabsize=2,
   	breaklines
}
\lstset{style=mystyle}
\chapter{Implementasi dan Pengujian}
\label{sec: Implementasi dan Pengujian}

Bab ini terdiri atas dua bagian, yaitu Implementasi Perangkat Lunak dan Pengujian Perangkat Lunak. Bagian implementasi berisi penjelasan bagaimana perangkat lunak dibuat dan langkah-langkah dalam pengunaan perangkat lunak. Sedangkan bagian pengujian berisi hasil pengujian fungsional terhadap perangkat lunak yang telah dibuat.
\section{Implementasi Perangkat Lunak}
\label{sec: Implementasi Perangkat Lunak}

Perangkat lunak dibuat menggunakan bahasa {\it java} dan dimasukkan ke dalam sebuah {\it jar} sehingga dapat digunakan dengan cara mengeksekusi perintah. Penggunaan {\it file jar} tersebut dapat dilakukan melalui {\it Terminal} di Linux/Mac atau {\it Command Prompt} di Windows. Berikut perintah yang digunakan untuk menjalankan perangkat lunak.
\\

{\tt javadoc [options] \{sourcefiles|packagenames\}}
\\

Pada potongan perintah diatas memiliki 2 parameter yaitu {\it option} dan {\it packagenames}. Parameter {\it packagename} adalah parameter untuk {\it package} yang akan sebagai masukan dari perangkat lunak. Parameter {\it option} adalah beberapa perintah pendukung. Berikut beberapa perintah {\it option} yang digunakan untuk mendukung berjalannya perangkat lunak.
\begin{itemize}
	\item {\tt -filename <file-name>} - Menghasilkan {\it output file} dengan nama {\tt file-name.tex}. Jika {\it option} ini tidak digunakan maka nama {\it file} yang dihasilkan akan bernama {\tt doc.tex}
	\item {\tt -doclet <class>} - Kelas yang dibuat untuk menghasilkan {\it output}.
	\item {\tt -docletpath <path>} - Letak {\it doclet} yang sudah di-{\it package} menjadi {\it file jar}
	\item {\tt -sourcepath <pathlist>} - Letak {\it source file} sebagai masukan.
	\item {\tt -subpackages <subpkglist>} - Letak {\it subpackage} yang akan dimuat secara rekursif.
\end{itemize}

Untuk penggunaan perintah diatas, Langkah pertama membuka aplikasi {\it Terminal} atau {\it Command Prompt}. Langkah kedua mengetik perintah {\tt javadoc} lalu diikuti dengan perintah pendukungnya seperti yang sudah dijelaskan diatas. Berikut contoh perintah lengkap yang digunakan.

\begin{verbatim}
	javadoc -filename <file-name>
	        -doclet extractor.Extractor
	        -docletpath GenerateJavadocToLatex.jar
	        -sourcepath <path/to/directory>
	        -subpackages <packagenames> <sourcefiles|packagenames>
\end{verbatim}

\section{Pengujian Perangkat Lunak}
\label{sec: pengujian perangkat lunak}
Pada sub bab ini akan menjelaskan Lingkungan Pengujian dan Pengujian Fungsional. Pengujian Fungsional akan menguji perangkat lunak terhadap kode program sederhana serta menguji kode program perangkat lunak yang dibuat.
\subsection{Lingkungan Pengujian}
\label{sec:lingkungan perangkat lunak}
Dalam proses pengujian perangkat lunak ini digunakan spesifikasi perangkat sebagai berikut.

\begin{enumerate}
	\item Processor: Intel Core i7 2.5-3.7GHz 
	\item RAM: 16.00 GB DDR3	
	\item Harddisk : 512MB SSD
	\item VGA : Intel Iris Pro dan AMD Radeon R9 M370X
	\item Sistem Operasi: macOS High Sierra
	\item Versi Java: 1.8.0\_121
	\item Code Editor: Netbeans 8.2
\end{enumerate}

\subsection{Pengujian Fungsional}
\label{sec:pengujian fungsional}
Pada pengujian fungsional dilakukan pengujian terhadap kode program sederhana dan kode program perangkat lunak yang dibuat. Berikut pengujian yang sudah dilakukan.
\begin{lstlisting}[language=TeX, caption=Hasil pengujian kode program sederhana]
\begin{enumerate}
\item \texttt{OperasiMatematikaInterface}

Kelas Abstract OperasiMatematika. Kelas ini memiliki method \texttt{calculate(int, int)}

Kelas ini tidak memiliki atribut. \textit{Method-method} yang dimiliki kelas ini adalah sebagai berikut.
\begin{itemize}
\item \texttt{public int calculate(int a, int b)}Method untuk menghasilkan perhitungan 2 buah bilangan

\textbf{Parameter:}
\begin{itemize}
\item \texttt{int a} - 
Bilangan pertama
\item \texttt{int b} - 
Bilagan kedua
\end{itemize}
\textbf{Return Value}: hasil perhitungan 2 buah bilangan  DoubleNaN

\textbf{Exception}: Tidak memiliki \textit{exception}

\end{itemize}
\item \texttt{Pembagian}

Kelas ini merupakan Kelas Pembagian

Atribut yang dimiliki kelas ini adalah sebagai berikut.
\begin{itemize}
\item \texttt{int a} - Atribut A
\item \texttt{int b} - Atribut B
\end{itemize}
\textit{Method-method} yang dimiliki kelas ini adalah sebagai berikut.
\begin{itemize}
\item \texttt{public int calculate(int a, int b)}Method untuk menghasilkan perhitungan 2 buah bilangan

\textbf{Parameter:}
\begin{itemize}
\item \texttt{int a} - 
Bilangan pertama
\item \texttt{int b} - 
Bilagan kedua
\end{itemize}
\textbf{Return Value}: hasil perhitungan 2 buah bilangan  DoubleNaN

\textbf{Exception}: Tidak memiliki \textit{exception}

\end{itemize}
\item \texttt{Pengurangan}

Kelas ini merupakan Kelas Pengurangan

Atribut yang dimiliki kelas ini adalah sebagai berikut.
\begin{itemize}
\item \texttt{int a} - Atribut A
\item \texttt{int b} - Atribut B
\end{itemize}
\textit{Method-method} yang dimiliki kelas ini adalah sebagai berikut.
\begin{itemize}
\item \texttt{public int calculate(int a, int b)}Method untuk menghasilkan perhitungan 2 buah bilangan

\textbf{Parameter:}
\begin{itemize}
\item \texttt{int a} - 
Bilangan pertama
\item \texttt{int b} - 
Bilagan kedua
\end{itemize}
\textbf{Return Value}: hasil perhitungan 2 buah bilangan  DoubleNaN

\textbf{Exception}: Tidak memiliki \textit{exception}

\end{itemize}
\item \texttt{Perkalian}

Kelas ini merupakan Kelas Perkalian

Atribut yang dimiliki kelas ini adalah sebagai berikut.
\begin{itemize}
\item \texttt{int a} - Atribut A
\item \texttt{int b} - Atribut B
\end{itemize}
\textit{Method-method} yang dimiliki kelas ini adalah sebagai berikut.
\begin{itemize}
\item \texttt{public int calculate(int a, int b)}Method untuk menghasilkan perhitungan 2 buah bilangan

\textbf{Parameter:}
\begin{itemize}
\item \texttt{int a} - 
Bilangan pertama
\item \texttt{int b} - 
Bilagan kedua
\end{itemize}
\textbf{Return Value}: hasil perhitungan 2 buah bilangan  DoubleNaN

\textbf{Exception}: Tidak memiliki \textit{exception}

\end{itemize}
\item \texttt{Pertambahan}

Kelas ini merupakan Kelas Pertambahan

Atribut yang dimiliki kelas ini adalah sebagai berikut.
\begin{itemize}
\item \texttt{int a} - Atribut A
\item \texttt{int b} - Atribut B
\end{itemize}
\textit{Method-method} yang dimiliki kelas ini adalah sebagai berikut.
\begin{itemize}
\item \texttt{public int calculate(int a, int b)}Method untuk menghasilkan perhitungan 2 buah bilangan

\textbf{Parameter:}
\begin{itemize}
\item \texttt{int a} - 
Bilangan pertama
\item \texttt{int b} - 
Bilagan kedua
\end{itemize}
\textbf{Return Value}: hasil perhitungan 2 buah bilangan  DoubleNaN

\textbf{Exception}: Tidak memiliki \textit{exception}

\end{itemize}
\end{enumerate}

\end{lstlisting}

\begin{lstlisting}[language=TeX, caption=Hasil Pengujian kode program perangkat lunak]
\begin{enumerate}
\item \texttt{AttributeClassExtractor}

Kelas ini merupakan kelas untuk mengambil informasi sebuah atribut yang
 terdapat pada kelas

Kelas ini tidak memiliki atribut. \textit{Method-method} yang dimiliki kelas ini adalah sebagai berikut.
\begin{itemize}
\item \texttt{public static void extractAttributeClassContent(FieldDoc[] fields, java.io.BufferedWriter out)}\textit{Method} ini akan menampilkan atribut-atribut yang dimiliki oleh
 sebuah kelas

\textbf{Parameter:}
\begin{itemize}
\item \texttt{FieldDoc fields} - 
sebuah array berisikan sejumlah atribut dari kelas
\item \texttt{BufferedWriter out} - 
turunan dari kelas \texttt{Writer} yang digunakan untuk menulis
 file text
\end{itemize}
\textbf{Return Value}: Tidak memiliki \textit{return value}

\textbf{Exception}: Tidak memiliki \textit{exception}

\end{itemize}
\item \texttt{ClassExtractor}

Kelas ini merupakan kelas untuk mengambil informasi dari sebuah kelas

Kelas ini tidak memiliki atribut. \textit{Method-method} yang dimiliki kelas ini adalah sebagai berikut.
\begin{itemize}
\item \texttt{public static void extractClassContent(ClassDoc[] classes, java.io.BufferedWriter out)}\textit{Method} ini akan menampilkan nama kelas berserta penjelasan dari
 sebuah kelas

\textbf{Parameter:}
\begin{itemize}
\item \texttt{ClassDoc classes} - 
sebuah array berisikan sejumlah kelas
\item \texttt{BufferedWriter out} - 
turunan dari kelas \texttt{Writer} yang digunakan untuk menulis
                file text
\end{itemize}
\textbf{Return Value}: Tidak memiliki \textit{return value}

\textbf{Exception}: Tidak memiliki \textit{exception}

\end{itemize}
\item \texttt{Extractor}

Kelas ini merupakan kelas untuk menjalan \textit{custom doclet}

Atribut yang dimiliki kelas ini adalah sebagai berikut.
\begin{itemize}
\item \texttt{String fileName} - atribut untuk nama \textit{file}
\end{itemize}
\textit{Method-method} yang dimiliki kelas ini adalah sebagai berikut.
\begin{itemize}
\item \texttt{public static boolean start(RootDoc root)}\textit{Method} ini berperan sebagai \textit{method} untuk menjalankan
 \textit{custom doclet}

\textbf{Parameter:}
\begin{itemize}
\item \texttt{RootDoc root} - 
berperan sebagai mengambil seluruh informasi spesifik dari
             \textit{option} yang terdapat pada \textit{command-line} sebuah
             \textit{terminal}. Selain itu berperan juga untuk mengambil informasi dari
             sekumpulan \textit{file java} yang akan di proses.
\end{itemize}
\textbf{Return Value}: kondisi true

\textbf{Exception}: Tidak memiliki \textit{exception}

\item \texttt{private static void init(ClassDoc[] classes)}\textit{Method} ini berperan untuk menulis kedalam sebuah \textit{file}
 saat \textit{javadoc} berjalan.

\textbf{Parameter:}
\begin{itemize}
\item \texttt{ClassDoc classes} - 
sebuah array yang berisikan sekumpulan \textit{file java}
                yang akan di proses.
\end{itemize}
\textbf{Return Value}: Tidak memiliki \textit{return value}

\textbf{Exception}: Tidak memiliki \textit{exception}

\item \texttt{public static int optionLength(java.lang.String option)}Method untuk menghitung banyak option yang digunakan pada
 \textit{command-line}

\textbf{Parameter:}
\begin{itemize}
\item \texttt{String option} - 
sebuah option
\end{itemize}
\textbf{Return Value}: panjang setiap option

\textbf{Exception}: Tidak memiliki \textit{exception}

\item \texttt{public static boolean validOptions(java.lang.String[][] args, DocErrorReporter err)}Pengecekan option valid

\textbf{Parameter:}
\begin{itemize}
\item \texttt{String args} - 
String array 2 dimensi dari option
\item \texttt{DocErrorReporter err} - 
sebuah error jika tidak terdapat option tersebut.
\end{itemize}
\textbf{Return Value}: bernilai true jika option tersebut dikenali, false jika option
 tersebut tidak dikenali

\textbf{Exception}: Tidak memiliki \textit{exception}

\end{itemize}
\item \texttt{MethodClassExtractor}

Kelas ini merupakan kelas untuk mengambil informasi sebuah \textit{method}
 terdapat pada kelas

Kelas ini tidak memiliki atribut. \textit{Method-method} yang dimiliki kelas ini adalah sebagai berikut.
\begin{itemize}
\item \texttt{public static void extractMethodClassContent(ClassDoc superclass, MethodDoc[] methods, java.io.BufferedWriter out)}\textit{Method} ini akan menampilkan \textit{method-method} yang dimiliki
 oleh sebuah kelas

\textbf{Parameter:}
\begin{itemize}
\item \texttt{ClassDoc superclass} - 
sebuah objek ClassDoc
\item \texttt{MethodDoc methods} - 
sebuah array berisikan sejumlah \textit{method} dari kelas
\item \texttt{BufferedWriter out} - 
turunan dari kelas \texttt{Writer} yang digunakan untuk menulis
                   file text
\end{itemize}
\textbf{Return Value}: Tidak memiliki \textit{return value}

\textbf{Exception}: Tidak memiliki \textit{exception}

\item \texttt{private static void ParameterMethod(java.io.BufferedWriter out, Parameter[] paramMethod, ParamTag[] paramTags)}\textit{Method} ini akan menampilkan parameter \textit{method-method} yang dimiliki
 oleh sebuah kelas

\textbf{Parameter:}
\begin{itemize}
\item \texttt{BufferedWriter out} - 
turunan dari kelas \texttt{Writer} yang digunakan untuk menulis file text
\item \texttt{Parameter paramMethod} - 
sebuah array berisikan sejumlah \textit{method} dari kelas
\item \texttt{ParamTag paramTags} - 
sebuah array berisikan sejumlah parameter \textit{method} dari kelas
\end{itemize}
\textbf{Return Value}: Tidak memiliki \textit{return value}

\textbf{Exception}: Tidak memiliki \textit{exception}

\item \texttt{private static void ReturnTypeMethod(java.io.BufferedWriter out, Type type, Tag[] returnTags)}\textit{Method} ini akan menampilkan \textit{return type} dari \textit{method-method} yang dimiliki
 oleh sebuah kelas

\textbf{Parameter:}
\begin{itemize}
\item \texttt{BufferedWriter out} - 
turunan dari kelas \texttt{Writer} yang digunakan untuk menulis file text
\item \texttt{Type type} - 
sebuah objek Type
\item \texttt{Tag returnTags} - 
sebuah array berisikan sejumlah \textit{return type} dari \textit{method} dari kelas
\end{itemize}
\textbf{Return Value}: Tidak memiliki \textit{return value}

\textbf{Exception}: Tidak memiliki \textit{exception}

\item \texttt{private static void ExceptionMethod(java.io.BufferedWriter out, Tag[] throwTags)}\textit{Method} ini akan menampilkan \textit{return type} dari \textit{method-method} yang dimiliki
 oleh sebuah kelas

\textbf{Parameter:}
\begin{itemize}
\item \texttt{BufferedWriter out} - 
turunan dari kelas \texttt{Writer} yang digunakan untuk menulis file text
\item \texttt{Tag throwTags} - 
sebuah array berisikan sejumlah \textit{exception} dari \textit{method} dari kelas
\end{itemize}
\textbf{Return Value}: Tidak memiliki \textit{return value}

\textbf{Exception}: Tidak memiliki \textit{exception}

\end{itemize}
\end{enumerate}
\end{lstlisting}

Hasil pengujian lengkap terdapat pada lampiran \ref{lamp:A} untuk pengujian kode program sederhana dan lampiran \ref{lamp:B} untuk pengujian kode program perangkat lunak.

\subsection{Pengujian Eksperimental}
\label{sec:pengujian eksperimental}
Pengujian eksperimental dilakukan terhadap kode program SIAModels. SIAModels adalah sekumpulan kelas {\it java} yang mewakili objek-objek Sistem Informasi Akademik Universitas Katolik Parahyangan. SIAModels ini berisikan objek matakuliah pada Fakultas Teknologi Informasi dan Sains.  Hasil pengujian terdapat pada lampiran \ref{lamp:C}























