%\lstdefinestyle{mystyle}{
%    basicstyle=\footnotesize,                
%   	captionpos=b,                     
%   	numbers=left,                    
%   	numbersep=5pt,             
%   	tabsize=2,
%   	breaklines
%}
%\lstset{style=mystyle}
\chapter{Implementasi dan Pengujian}
\label{sec: Implementasi dan Pengujian}

Bab ini terdiri atas dua bagian, yaitu Implementasi Perangkat Lunak dan Pengujian Perangkat Lunak. Bagian implementasi berisi penjelasan bagaimana perangkat lunak dibuat dan langkah-langkah dalam pengunaan perangkat lunak. Sedangkan bagian pengujian berisi hasil pengujian fungsional terhadap perangkat lunak yang telah dibuat.
\section{Implementasi Perangkat Lunak}
\label{sec: Implementasi Perangkat Lunak}

Perangkat lunak dibuat menggunakan bahasa {\it java} dan dimasukkan ke dalam sebuah {\it jar} sehingga dapat digunakan dengan cara mengeksekusi perintah. Penggunaan {\it file jar} tersebut dapat dilakukan melalui {\it Terminal} di Linux/Mac atau {\it Command Prompt} di Windows. Berikut perintah yang digunakan untuk menjalankan perangkat lunak.
\begin{verbatim}
	javadoc -filename <file-name>
	        -doclet extractor.Extractor
	        -docletpath GenerateJavadocToLatex.jar
	        -sourcepath <path/to/directory>
	        -subpackages <packagenames>
	        <sourcefiles|packagenames>
\end{verbatim}
Pada potongan perintah diatas memiliki 2 parameter yaitu {\it option} dan {\it packagenames}. Parameter {\it packagename} adalah parameter untuk {\it package} yang akan sebagai masukan dari perangkat lunak. Parameter {\it option} adalah beberapa perintah pendukung. Berikut beberapa perintah {\it option} yang digunakan untuk mendukung berjalannya perangkat lunak.
\begin{itemize}
	\item {\tt -filename <file-name>} - Menghasilkan {\it output file} dengan nama {\tt file-name.tex}. Jika {\it option} ini tidak digunakan maka nama {\it file} yang dihasilkan akan bernama {\tt doc.tex}
	\item {\tt -doclet <class>} - Kelas yang dibuat untuk menghasilkan {\it output}. Pada kasus ini adalah {\tt extractor.Extractor}
	\item {\tt -docletpath <path>} - Letak {\it doclet} yang sudah di-{\it package} menjadi {\it file jar}
	\item {\tt -sourcepath <pathlist>} - Letak {\it source file} sebagai masukan. Perintah ini dapat digunakan secara opsional jika masukan adalah sebuah {\it package}
	\item {\tt -subpackages <subpkglist>} - Letak {\it subpackage} yang akan dimuat secara rekursif. Perintah ini dapat digunakan secara opsional jika terdapat sebuah {\it subpackage} didalam {\it package} yang menjadi masukan.
	\item {\tt <sourcefiles|packagenames>} - Letak {\it sourcefiles} atau {\it packagenames} yang menjadi masukan. Jika masukan adalah sebuah {\tt packagenames} maka harus terdapat perintah {\tt -sourcepath} sebagai letak {\it packagenames} tersebut. Jika masukan adalah sebuah {\tt sourcefiles} maka tidak perlu terdapat perintah {\tt -sourcepath} karena masukan dapat berupa direktori {\it source files} tersebut.
\end{itemize}

Untuk penggunaan perintah diatas, Langkah pertama membuka aplikasi {\it Terminal} atau {\it Command Prompt}. Langkah kedua mengetik perintah {\tt javadoc} lalu diikuti dengan perintah pendukungnya seperti yang sudah dijelaskan diatas. Berikut contoh perintah lengkap yang digunakan.

\begin{verbatim}
	javadoc -filename siamodels
	        -doclet extractor.Extractor
	        -docletpath GenerateJavadocToLatex.jar
	        -sourcepath SIAModels/src/main/java
	        -subpackages id.ac.unpar.siamodels
	        id.ac.unpar.siamodels
\end{verbatim}

\section{Pengujian Perangkat Lunak}
\label{sec: pengujian perangkat lunak}
Pada sub bab ini akan menjelaskan Lingkungan Pengujian dan Pengujian Fungsional. Pengujian Fungsional akan menguji perangkat lunak terhadap kode program sederhana serta menguji kode program perangkat lunak yang dibuat.
\subsection{Lingkungan Pengujian}
\label{sec:lingkungan perangkat lunak}
Dalam proses pengujian perangkat lunak ini digunakan spesifikasi perangkat sebagai berikut.

\begin{enumerate}
	\item Processor: Intel Core i7 2.5-3.7GHz 
	\item RAM: 16.00 GB DDR3	
	\item Harddisk : 512MB SSD
	\item VGA : Intel Iris Pro dan AMD Radeon R9 M370X
	\item Sistem Operasi: macOS High Sierra
	\item Versi Java: 1.8.0\_121
	\item Code Editor: Netbeans 8.2
\end{enumerate}

\subsection{Pengujian Fungsional}
\label{sec:pengujian fungsional}
Pada pengujian fungsional dilakukan terhadap 2 kasus yaitu pengujian kode program sederhana dan kode program perangkat lunak yang dibuat.

Pada kasus pertama dijalankan perintah sebagai berikut.

\begin{verbatim}
	javadoc -filename operasimatematika
	        -doclet extractor.Extractor
	        -docletpath GenerateJavadocToLatex.jar
	        -sourcepath OperasiMatematika/src
	        operasi
\end{verbatim}

Perintah tersebut mengekstrak kode program yang dapat dilihat pada lampiran \ref{kodeProgram:A}. Setelah perintah tersebut dieksekusi akan menghasilkan {\it file} \LaTeX\ yang dapat dilihat pada lampiran \ref{hasilLatex:A}. Jika {\it file} \LaTeX\ tersebut diekseskusi menggunakan perintah {\tt pdflatex} akan menghasil sebuah {\it file} PDF yang dapat dilihat pada lampiran \ref{hasilPDF:A}.

Dapat dilihat dari hasil \ref{hasilPDF:A} terdapat 5 kelas yang diekstrak menjadi 5 buah {\it item list enumarate} yang masing berisikan nama kelas dari 5 file tersebut. {\it Method-method} yang terdapat kelas tersebut akan dibuat dengan menggunakan {\it list itemize} sebanyak {\tt n} buah {\it method} dan parameter dari {\it method} tersebut dibuat menggunakan {\it list itemize} sebanyak {\tt n} buah parameter.

Pada kasus kedua dijalankan perintah sebagai berikut.

\begin{verbatim}
	javadoc -filename javadoctolatex
	        -doclet extractor.Extractor
	        -docletpath GenerateJavadocToLatex.jar
	        -sourcepath GenerateJavadocToLatex/src
	        extractor
\end{verbatim}

Perintah tersebut mengekstrak kode program yang dapat dilihat pada lampiran \ref{kodeProgram:B}. Setelah perintah tersebut dieksekusi akan menghasilkan {\it file} \LaTeX\ yang dapat dilihat pada lampiran \ref{hasilLatex:B}. Jika {\it file} \LaTeX\ tersebut diekseskusi menggunakan perintah {\tt pdflatex} akan menghasil sebuah {\it file} PDF yang dapat dilihat pada lampiran \ref{hasilPDF:B}.

Dapat dilihat dari hasil \ref{hasilPDF:B} terdapat 4 kelas yang diekstrak menjadi 4 buah {\it item list enumarate} yang masing berisikan nama kelas dari 4 file tersebut. Atribut kelas akan dibuat dengan menggunakan {\it list itemize} sebanyak {\tt n} buah atribut. {\it Method-method} yang terdapat kelas tersebut akan dibuat dengan menggunakan {\it list itemize} sebanyak {\tt n} buah {\it method} dan parameter dari {\it method} tersebut dibuat menggunakan {\it list itemize} sebanyak {\tt n} buah parameter.

\subsection{Pengujian Eksperimental}
\label{sec:pengujian eksperimental}
Pengujian eksperimental dilakukan terhadap kode program SIAModels. SIAModels adalah sekumpulan kelas {\it java} yang mewakili objek-objek Sistem Informasi Akademik Universitas Katolik Parahyangan. SIAModels ini berisikan objek matakuliah pada Fakultas Teknologi Informasi dan Sains.

Pada pengujian menggunakan SIAModels dilakukan perintah sebagai berikut.

\begin{verbatim}
	javadoc -filename siamodels
	        -doclet extractor.Extractor
	        -docletpath GenerateJavadocToLatex.jar
	        -sourcepath SIAModels/src/main/java
	        -subpackages id.ac.unpar.siamodels
	        id.ac.unpar.siamodels
\end{verbatim}

Perintah tersebut mengekstrak kode program yang dapat dilihat pada lampiran \ref{kodeProgram:C}. Setelah perintah tersebut dieksekusi akan menghasilkan {\it file} \LaTeX\ yang dapat dilihat pada lampiran \ref{hasilLatex:C}. Jika {\it file} \LaTeX\ tersebut diekseskusi menggunakan perintah {\tt pdflatex} akan menghasil sebuah {\it file} PDF yang dapat dilihat pada lampiran \ref{hasilPDF:C}.

Dapat dilihat dari hasil \ref{hasilPDF:C} terdapat {\tt n} buah kelas dan {\tt n} buah {\it innerclass} yang diekstrak menjadi {\tt n} buah {\it item list enumarate} yang masing berisikan nama kelas dan {\it innerclass} dari {\tt n} {\it file} tersebut. Atribut kelas akan dibuat dengan menggunakan {\it list itemize} sebanyak {\tt n} buah atribut. {\it Method-method} yang terdapat kelas tersebut akan dibuat dengan menggunakan {\it list itemize} sebanyak {\tt n} buah {\it method} dan parameter dari {\it method} tersebut dibuat menggunakan {\it list itemize} sebanyak {\tt n} buah parameter. Terdapat beberapa ketidaklazimanan yang ditemukan setelah mengeksekusi perintah {\tt pdflatex}.

\begin{itemize}
	\item Terdapat karakter yang tidak lazim yaitu tanda \textexclamdown\ yang merepresentasikan tanda lebih kecil "<" dan tanda \textquestiondown\ yang merepresentasikan tanda lebih besar ">". Karakter tersebut muncul pada kelas: 
		\begin{enumerate}
			\item {\tt TahunSemester} - pada atribut kelas
			\item {\tt Mahasiswa} - pada {\it method} {\tt calculateIPKLulus()}, {\tt calculateIPLulus()}, {\tt calculateIPTempuh()}, {\tt calculateIPKumulatif()}, {\tt calculateIPKTempuh()} dan {\tt calculateIPS()}
		\end{enumerate}
		Karakter yang tidak lazim tersebut muncul karena {\it file} \LaTeX\ yang dibuat oleh perangkat lunak tidak menggunakan {\it package} {\tt \string\usepackage[T1]\{fontenc\}} atau merubahnya menjadi command {\tt \string\textless} sebagai tanda lebih kecil dan {\tt \string\textgreater} sebagai tanda lebih besar. Pada \LaTeX\ beberapa spesial karakter dapat dimunculkan dengan menambahkan tanda {\it backslash} sebelum karakternya seperti karakter \_ {\it (underscore)} pada \LaTeX\ harus ditulis seperti \texttt{\string\_} dan beberapa spesial karakter yang tidak dapat dimunculkan dengan menambahkan tanda {\it backslash} harus menambahkan {\it package} tersebut.
	\item Beberapa deskripsi tidak muncul pada hasil PDF, contohnya seperti pada kelas:
		\begin{enumerate}
			\item {\tt Semester} - pada deskripsi kelas, atribut kelas dan {\it method}
			\item {\tt TahunSemester} - pada deskripsi {\it method} seperti pada {\it method} {\tt  getTahun()}
			\item {\tt Mahasiswa} - pada deskripsi kelas, atribut kelas dan beberapa {\it method}
			\item {\tt Mahasiswa.Nilai} - pada beberapa deskripsi {\it method}
			\item {\tt JadwalKuliah} - pada deskrpisi kelas, atribut kelas dan {\it method}
			\item {\tt MataKuliah} - pada deskripsi kelas, atribut kelas, dan {\it method}
			\item {\tt Dosen} - pada deskripsi kelas, atribut kelas dan {\it method}
		\end{enumerate}
		Setelah dilakukan pemeriksaan pada hal tersebut, kesalahan yang terjadi bukan dari perangkat lunak melainkan kode program yang menjadi masukan memiliki dokumentasi yang kurang lengkap.
	\item Parameter setiap {\it method} memiliki tipe variabel yang panjang. Setelah dilakukan pemeriksaan, tipe variabel yang bersifat Objek akan menghasilkan tipe variabel yang {\it fullqualified} seperti contohnya pada tipe variabel {\tt String} yang akan menghasilkan {\tt java.lang.String}.
\end{itemize}






















