\documentclass{article}
\begin{document}
\begin{enumerate}
\item \texttt{HasPraktikum}\\ 
Kelas ini tidak memiliki atribut. \textit{Method-method} yang dimiliki kelas ini adalah sebagai berikut.
\begin{itemize}
\end{itemize}
\item \texttt{HasPrasyarat}\\ 
Mendefinisikan kelas-kelas yang memiliki prasyarat, terkustomisasi untuk seorang {@link Mahasiswa}. Jika ada tambahan, jangan lupa untuk mendaftarkannya di {@link #DEFAULT_HASPRASYARAT_CLASSES}. Jika berubah package, jangan lupa mengupdate {@link #DEFAULT_PRASYARAT_PACKAGE}..

Atribut yang dimiliki kelas ini adalah sebagai berikut.
\begin{itemize}
\item \texttt{String DEFAULT_HASPRASYARAT_CLASSES} - Daftar dari nama kelas default seluruh turunan interface ini. Perlu didaftarkan
 manual, karena Java reflection tidak dapat mendeteksi otomatis.
\item \texttt{String DEFAULT_PRASYARAT_PACKAGE} - Package tempat menyimpan seluruh turunan standar interface ini. Perlu didefinisikan
 manual, karena Java reflection tidak dapat mendeteksi otomatis.
\end{itemize}
\textit{Method-method} yang dimiliki kelas ini adalah sebagai berikut.
\begin{itemize}
\item \texttt{public boolean checkPrasyarat(Mahasiswa mahasiswa, java.util.List reasonsContainer)}\\ 
Memeriksa prasyarat-prasyarat dari kuliah, spesifik untuk mahasiswa
 yang dituju. Jika ada pesan-pesan khusus, akan ditambahkan pada parameter
 reasonsContainer.

\textbf{Parameter:}
\begin{itemize}
\item \texttt{Mahasiswa mahasiswa} - 
prasyarat kuliah akan diperiksa spesifik pada mahasiswa ini
\item \texttt{java.util.List reasonsContainer} - 
pesan-pesan terkait prasyarat akan ditambahkan di sini, jika ada.
\end{itemize}
\textbf{Kembalian}: true jika seluruh prasyarat dipenuhi, false jika tidak.

\textbf{Exception}: Tidak memiliki \textit{exception}

\end{itemize}
\item \texttt{HasResponsi}\\ 
.

Kelas ini tidak memiliki atribut. \textit{Method-method} yang dimiliki kelas ini adalah sebagai berikut.
\begin{itemize}
\end{itemize}
\item \texttt{Kelulusan}\\ 
.

Atribut yang dimiliki kelas ini adalah sebagai berikut.
\begin{itemize}
\item \texttt{String PILIHAN_WAJIB} - 
\item \texttt{String WAJIB} - 
\item \texttt{String AGAMA} - 
\item \texttt{int MIN_SKS_LULUS} - 
\item \texttt{int MIN_PILIHAN_WAJIB} - 
\end{itemize}
\textit{Method-method} yang dimiliki kelas ini adalah sebagai berikut.
\begin{itemize}
\item \texttt{public boolean checkPrasyarat(Mahasiswa mahasiswa, java.util.List reasonsContainer)}\\ 


\textbf{Parameter:}\begin{itemize}
\item Tidak memiliki parameter \textit{method}
\end{itemize}
\textbf{Kembalian}: Tidak memiliki \textit{return value}

\textbf{Exception}: Tidak memiliki \textit{exception}

\textbf{Override}: \texttt{checkPrasyarat} dari kelas \texttt{Object}

\end{itemize}
\end{enumerate}
\end{document}
