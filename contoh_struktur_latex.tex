\documentclass{article}
\title{Javadoc}
\author{Adli Fariz Bonaputra \\ Teknik Informatika Unpar}
\begin{document}
\maketitle
\begin{enumerate}
	\item \texttt{Pertambahan}\\
	Kelas ini merupakan Kelas Pertambahan.
	
	Atribut yang dimiliki kelas ini adalah sebagai berikut.
	\begin{itemize}
		\item \texttt{int a} -
		{penjelasan tentang atribut a}.
		\item \texttt{int b} -
		{penjelasan tentang atribut b}.
	\end{itemize}
	
	{\it Method} yang terdapat pada kelas Pertambahan adalah sebagai berikut.
	\begin{itemize}
		\item \texttt{public int add(int a, int b)}\\
		{penjelasan tentang method add()}.
		
		\textbf{Parameter:}
		\begin{itemize}
			\item \texttt{int a} - 
			{penjelasan dari parameter a}.
			\item \texttt{int b} - 
			{penjelasan dari parameter b}.
		\end{itemize}
		
		\textbf{Return Value:} {penjelasan return-type method}.\\
		\textbf{Exception:} {penjelasan exception jika terdapat exception}.
	\end{itemize}
\end{enumerate}

\end{document}