\documentclass{article}
\title{\textbf{Konversi Javadoc ke \LaTeX}}
\author{Adli Fariz Bonaputra \\ Teknik Informatika Unpar}
\begin{document}
\maketitle
\section{Diagram Kelas Rinci}
\begin{enumerate}
\item \texttt{Pembagian}\\ 
Kelas ini merupakan Kelas Pembagian

Atribut yang dimiliki kelas ini adalah sebagai berikut.
\begin{itemize}
\item Tidak memiliki Atribut
\end{itemize}
\textit{Method-method} yang dimiliki kelas ini adalah sebagai berikut.
\begin{itemize}
\item \texttt{public void Pembagian(int a, int b)}\\ 
yaya

\textbf{Parameter:}
\begin{itemize}
\item \texttt{int a} - 
bilangan 1
\item \texttt{int b} - 
bilangan 2
\end{itemize}
\textbf{Kembalian}: Tidak memiliki \textit{return value}

\textbf{Exception}: Tidak memiliki \textit{exception}

\item \texttt{public int pembagian(int a, int b)}\\ 
Method Pembagian

\textbf{Parameter:}
\begin{itemize}
\item \texttt{int a} - 
Bilangan Pertama
\item \texttt{int b} - 
Bilangan Kedua
\end{itemize}
\textbf{Kembalian}: hasil pembagian 2 buah bilangan

\textbf{Exception}: Tidak memiliki \textit{exception}

\end{itemize}
\item \texttt{Pengurangan}\\ 
Kelas ini merupakan Kelas Pengurangan

Atribut yang dimiliki kelas ini adalah sebagai berikut.
\begin{itemize}
\item \texttt{int a} - Atribut A
\item \texttt{int b} - Atribut B
\end{itemize}
\textit{Method-method} yang dimiliki kelas ini adalah sebagai berikut.
\begin{itemize}
\item \texttt{public int pengurangan(int a, int b)}\\ 
Method Pengurangan

\textbf{Parameter:}
\begin{itemize}
\item \texttt{int a} - 
Bilangan Pertama
\item \texttt{int b} - 
Bilangan Kedua
\end{itemize}
\textbf{Kembalian}: hasil pengurangan 2 buah bilangan

\textbf{Exception}: Tidak memiliki \textit{exception}

\textbf{Override}: \texttt{pengurangan} dari kelas \texttt{Pengurangan}

\end{itemize}
\item \texttt{Perkalian}\\ 
Kelas ini merupakan Kelas Perkalian

Atribut yang dimiliki kelas ini adalah sebagai berikut.
\begin{itemize}
\item \texttt{int a} - Atribut A
\item \texttt{int b} - Atribut B
\end{itemize}
\textit{Method-method} yang dimiliki kelas ini adalah sebagai berikut.
\begin{itemize}
\item \texttt{public int perkalian(int a, int b)}\\ 
Method Perkalian

\textbf{Parameter:}
\begin{itemize}
\item \texttt{int a} - 
Bilangan Pertama
\item \texttt{int b} - 
Bilangan Kedua
\end{itemize}
\textbf{Kembalian}: hasil perkalian 2 buah bilangan

\textbf{Exception}: Tidak memiliki \textit{exception}

\end{itemize}
\item \texttt{Pertambahan}\\ 
Kelas ini merupakan Kelas Pertambahan kelas ini memiliki beberapa fungsi
 yaitu {@link #pertambahan(int,int) pertambahan}

Atribut yang dimiliki kelas ini adalah sebagai berikut.
\begin{itemize}
\item \texttt{int a} - Atribut A
\item \texttt{int b} - Atribut B
\end{itemize}
\textit{Method-method} yang dimiliki kelas ini adalah sebagai berikut.
\begin{itemize}
\item \texttt{public int pertambahan(int a, int b)}\\ 
Method Pertambahan

\textbf{Parameter:}
\begin{itemize}
\item \texttt{int a} - 
Bilangan Pertama
\item \texttt{int b} - 
Bilangan Kedua
\end{itemize}
\textbf{Kembalian}: hasil penjumlahan 2 buah bilangan

\textbf{Exception}: Tidak memiliki \textit{exception}

\textbf{Override}: \texttt{pertambahan} dari kelas \texttt{Pertambahan}

\end{itemize}
\item \texttt{operasiMatematikaInterface}\\ 


Atribut yang dimiliki kelas ini adalah sebagai berikut.
\begin{itemize}
\item Tidak memiliki Atribut
\end{itemize}
\textit{Method-method} yang dimiliki kelas ini adalah sebagai berikut.
\begin{itemize}
\item \texttt{public int pertambahan(int a, int b)}\\ 


\textbf{Kembalian}: Tidak memiliki \textit{return value}

\textbf{Exception}: Tidak memiliki \textit{exception}

\item \texttt{public int pengurangan(int a, int b)}\\ 


\textbf{Kembalian}: Tidak memiliki \textit{return value}

\textbf{Exception}: Tidak memiliki \textit{exception}

\item \texttt{public int pembagian(int a, int b)}\\ 


\textbf{Kembalian}: Tidak memiliki \textit{return value}

\textbf{Exception}: Tidak memiliki \textit{exception}

\item \texttt{public int perkalian(int a, int b)}\\ 


\textbf{Kembalian}: Tidak memiliki \textit{return value}

\textbf{Exception}: Tidak memiliki \textit{exception}

\end{itemize}
\end{enumerate}
\end{document}
