\documentclass{article}
\begin{document}
\begin{enumerate}
\item \texttt{Perkalian}\\ 
Kelas ini merupakan Kelas Perkalian \texttt{calculate(int, int)}

Atribut yang dimiliki kelas ini adalah sebagai berikut.
\begin{itemize}
\item \texttt{int a} - Atribut A
\item \texttt{int b} - Atribut B
\end{itemize}
\textit{Method-method} yang dimiliki kelas ini adalah sebagai berikut.
\begin{itemize}
\item \texttt{public int calculate(int a, int b)}\\ 
Method untuk menghasilkan perhitungan 2 buah bilangan \texttt{calculate(int, int)}

\textbf{Parameter:}
\begin{itemize}
\item \texttt{int a} - 
Bilangan pertama
\item \texttt{int b} - 
Bilagan kedua
\end{itemize}
\textbf{Return Value}: hasil perhitungan 2 buah bilangan {#link DoubleNaN}

\textbf{Exception}: Tidak memiliki \textit{exception}

\end{itemize}
\item \texttt{Pengurangan}\\ 
Kelas ini merupakan Kelas Pengurangan \texttt{calculate(int, int)}

Atribut yang dimiliki kelas ini adalah sebagai berikut.
\begin{itemize}
\item \texttt{int a} - Atribut A
\item \texttt{int b} - Atribut B
\end{itemize}
\textit{Method-method} yang dimiliki kelas ini adalah sebagai berikut.
\begin{itemize}
\item \texttt{public int calculate(int a, int b)}\\ 
Method untuk menghasilkan perhitungan 2 buah bilangan \texttt{calculate(int, int)}

\textbf{Parameter:}
\begin{itemize}
\item \texttt{int a} - 
Bilangan pertama
\item \texttt{int b} - 
Bilagan kedua
\end{itemize}
\textbf{Return Value}: hasil perhitungan 2 buah bilangan {#link DoubleNaN}

\textbf{Exception}: Tidak memiliki \textit{exception}

\end{itemize}
\item \texttt{Pembagian}\\ 
Kelas ini merupakan Kelas Pembagian \texttt{calculate(int, int)}

Atribut yang dimiliki kelas ini adalah sebagai berikut.
\begin{itemize}
\item \texttt{int a} - Atribut A
\item \texttt{int b} - Atribut B
\end{itemize}
\textit{Method-method} yang dimiliki kelas ini adalah sebagai berikut.
\begin{itemize}
\item \texttt{public int calculate(int a, int b)}\\ 
Method untuk menghasilkan perhitungan 2 buah bilangan \texttt{calculate(int, int)}

\textbf{Parameter:}
\begin{itemize}
\item \texttt{int a} - 
Bilangan pertama
\item \texttt{int b} - 
Bilagan kedua
\end{itemize}
\textbf{Return Value}: hasil perhitungan 2 buah bilangan {#link DoubleNaN}

\textbf{Exception}: Tidak memiliki \textit{exception}

\end{itemize}
\item \texttt{OperasiMatematikaInterface}\\ 
Kelas Abstract OperasiMatematika. Kelas ini memiliki method \texttt{calculate(int, int)}

Kelas ini tidak memiliki atribut. \textit{Method-method} yang dimiliki kelas ini adalah sebagai berikut.
\begin{itemize}
\item \texttt{public int calculate(int a, int b)}\\ 
Method untuk menghasilkan perhitungan 2 buah bilangan \texttt{calculate(int, int)}

\textbf{Parameter:}
\begin{itemize}
\item \texttt{int a} - 
Bilangan pertama
\item \texttt{int b} - 
Bilagan kedua
\end{itemize}
\textbf{Return Value}: hasil perhitungan 2 buah bilangan {#link DoubleNaN}

\textbf{Exception}: Tidak memiliki \textit{exception}

\end{itemize}
\item \texttt{Pertambahan}\\ 
Kelas ini merupakan Kelas Pertambahan \texttt{calculate(int, int)}

Atribut yang dimiliki kelas ini adalah sebagai berikut.
\begin{itemize}
\item \texttt{int a} - Atribut A
\item \texttt{int b} - Atribut B
\end{itemize}
\textit{Method-method} yang dimiliki kelas ini adalah sebagai berikut.
\begin{itemize}
\item \texttt{public int calculate(int a, int b)}\\ 
Method untuk menghasilkan perhitungan 2 buah bilangan \texttt{calculate(int, int)}

\textbf{Parameter:}
\begin{itemize}
\item \texttt{int a} - 
Bilangan pertama
\item \texttt{int b} - 
Bilagan kedua
\end{itemize}
\textbf{Return Value}: hasil perhitungan 2 buah bilangan {#link DoubleNaN}

\textbf{Exception}: Tidak memiliki \textit{exception}

\end{itemize}
\end{enumerate}
\end{document}
