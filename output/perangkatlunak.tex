\documentclass{article}
\begin{document}
\begin{enumerate}
\item \texttt{Extractor}

Kelas ini merupakan kelas untuk menjalan \textit{custom doclet}

Atribut yang dimiliki kelas ini adalah sebagai berikut.
\begin{itemize}
\item \texttt{String fileName} - atribut untuk nama \textit{file}
\end{itemize}
\textit{Method-method} yang dimiliki kelas ini adalah sebagai berikut.
\begin{itemize}
\item \texttt{public static boolean start(com.sun.javadoc.RootDoc root)}

\textit{Method} ini berperan sebagai \textit{method} untuk menjalankan
 \textit{custom doclet}

\textbf{Parameter:}
\begin{itemize}
\item \texttt{RootDoc root} - 
berperan sebagai mengambil seluruh informasi spesifik dari
             \textit{option} yang terdapat pada \textit{command-line} sebuah
             \textit{terminal}. Selain itu berperan juga untuk mengambil informasi dari
             sekumpulan \textit{file java} yang akan di proses.
\end{itemize}
\textbf{Return Value}: kondisi true

\textbf{Exception}: Tidak memiliki \textit{exception}

\item \texttt{private static void init(com.sun.javadoc.ClassDoc[] classes)}

\textit{Method} ini berperan untuk menulis kedalam sebuah \textit{file}
 saat \textit{javadoc} berjalan.

\textbf{Parameter:}
\begin{itemize}
\item \texttt{ClassDoc[] classes} - 
sebuah array yang berisikan sekumpulan \textit{file java}
                yang akan di proses.
\end{itemize}
\textbf{Return Value}: Tidak memiliki \textit{return value}

\textbf{Exception}: Tidak memiliki \textit{exception}

\item \texttt{public static int optionLength(java.lang.String option)}

Method untuk menghitung banyak option yang digunakan pada
 \textit{command-line}

\textbf{Parameter:}
\begin{itemize}
\item \texttt{String option} - 
sebuah option
\end{itemize}
\textbf{Return Value}: panjang setiap option

\textbf{Exception}: Tidak memiliki \textit{exception}

\item \texttt{public static boolean validOptions(java.lang.String[][] args, com.sun.javadoc.DocErrorReporter err)}

Pengecekan option valid

\textbf{Parameter:}
\begin{itemize}
\item \texttt{String[][] args} - 
String array 2 dimensi dari option
\item \texttt{DocErrorReporter err} - 
sebuah error jika tidak terdapat option tersebut.
\end{itemize}
\textbf{Return Value}: bernilai true jika option tersebut dikenali, false jika option
 tersebut tidak dikenali

\textbf{Exception}: Tidak memiliki \textit{exception}

\end{itemize}
\item \texttt{ClassExtractor}

Kelas ini merupakan kelas untuk mengambil informasi dari sebuah kelas

Kelas ini tidak memiliki atribut. \textit{Method-method} yang dimiliki kelas ini adalah sebagai berikut.
\begin{itemize}
\item \texttt{public static void extractClassContent(com.sun.javadoc.ClassDoc[] classes, java.io.BufferedWriter out)}

\textit{Method} ini akan menampilkan nama kelas berserta penjelasan dari
 sebuah kelas

\textbf{Parameter:}
\begin{itemize}
\item \texttt{ClassDoc[] classes} - 
sebuah array berisikan sejumlah kelas
\item \texttt{BufferedWriter out} - 
turunan dari kelas \texttt{Writer} yang digunakan untuk menulis
                file text
\end{itemize}
\textbf{Return Value}: Tidak memiliki \textit{return value}

\textbf{Exception}: Tidak memiliki \textit{exception}

\end{itemize}
\item \texttt{MethodClassExtractor}

Kelas ini merupakan kelas untuk mengambil informasi sebuah \textit{method}
 terdapat pada kelas

Kelas ini tidak memiliki atribut. \textit{Method-method} yang dimiliki kelas ini adalah sebagai berikut.
\begin{itemize}
\item \texttt{public static void extractMethodClassContent(com.sun.javadoc.ClassDoc superclass, com.sun.javadoc.MethodDoc[] methods, java.io.BufferedWriter out)}

\textit{Method} ini akan menampilkan \textit{method-method} yang dimiliki
 oleh sebuah kelas

\textbf{Parameter:}
\begin{itemize}
\item \texttt{ClassDoc superclass} - 
sebuah objek ClassDoc
\item \texttt{MethodDoc[] methods} - 
sebuah array berisikan sejumlah \textit{method} dari kelas
\item \texttt{BufferedWriter out} - 
turunan dari kelas \texttt{Writer} yang digunakan untuk menulis
                   file text
\end{itemize}
\textbf{Return Value}: Tidak memiliki \textit{return value}

\textbf{Exception}: Tidak memiliki \textit{exception}

\item \texttt{private static void ParameterMethod(java.io.BufferedWriter out, com.sun.javadoc.Parameter[] paramMethod, com.sun.javadoc.ParamTag[] paramTags)}

\textit{Method} ini akan menampilkan parameter \textit{method-method} yang dimiliki
 oleh sebuah kelas

\textbf{Parameter:}
\begin{itemize}
\item \texttt{BufferedWriter out} - 
turunan dari kelas \texttt{Writer} yang digunakan untuk menulis file text
\item \texttt{Parameter[] paramMethod} - 
sebuah array berisikan sejumlah \textit{method} dari kelas
\item \texttt{ParamTag[] paramTags} - 
sebuah array berisikan sejumlah parameter \textit{method} dari kelas
\end{itemize}
\textbf{Return Value}: Tidak memiliki \textit{return value}

\textbf{Exception}: Tidak memiliki \textit{exception}

\item \texttt{private static void ReturnTypeMethod(java.io.BufferedWriter out, com.sun.javadoc.Type type, com.sun.javadoc.Tag[] returnTags)}

\textit{Method} ini akan menampilkan \textit{return type} dari \textit{method-method} yang dimiliki
 oleh sebuah kelas

\textbf{Parameter:}
\begin{itemize}
\item \texttt{BufferedWriter out} - 
turunan dari kelas \texttt{Writer} yang digunakan untuk menulis file text
\item \texttt{Type type} - 
sebuah objek Type
\item \texttt{Tag[] returnTags} - 
sebuah array berisikan sejumlah \textit{return type} dari \textit{method} dari kelas
\end{itemize}
\textbf{Return Value}: Tidak memiliki \textit{return value}

\textbf{Exception}: Tidak memiliki \textit{exception}

\item \texttt{private static void ExceptionMethod(java.io.BufferedWriter out, com.sun.javadoc.Tag[] throwTags)}

\textit{Method} ini akan menampilkan \textit{return type} dari \textit{method-method} yang dimiliki
 oleh sebuah kelas

\textbf{Parameter:}
\begin{itemize}
\item \texttt{BufferedWriter out} - 
turunan dari kelas \texttt{Writer} yang digunakan untuk menulis file text
\item \texttt{Tag[] throwTags} - 
sebuah array berisikan sejumlah \textit{exception} dari \textit{method} dari kelas
\end{itemize}
\textbf{Return Value}: Tidak memiliki \textit{return value}

\textbf{Exception}: Tidak memiliki \textit{exception}

\end{itemize}
\item \texttt{AttributeClassExtractor}

Kelas ini merupakan kelas untuk mengambil informasi sebuah atribut yang
 terdapat pada kelas

Kelas ini tidak memiliki atribut. \textit{Method-method} yang dimiliki kelas ini adalah sebagai berikut.
\begin{itemize}
\item \texttt{public static void extractAttributeClassContent(com.sun.javadoc.FieldDoc[] fields, java.io.BufferedWriter out)}

\textit{Method} ini akan menampilkan atribut-atribut yang dimiliki oleh
 sebuah kelas

\textbf{Parameter:}
\begin{itemize}
\item \texttt{FieldDoc[] fields} - 
sebuah array berisikan sejumlah atribut dari kelas
\item \texttt{BufferedWriter out} - 
turunan dari kelas \texttt{Writer} yang digunakan untuk menulis
 file text
\end{itemize}
\textbf{Return Value}: Tidak memiliki \textit{return value}

\textbf{Exception}: Tidak memiliki \textit{exception}

\end{itemize}
\end{enumerate}
\end{document}
