\documentclass{article}
\begin{document}
\begin{enumerate}
\item \texttt{InfoMataKuliah}\\ 
Kelas ini tidak memiliki atribut. \textit{Method-method} yang dimiliki kelas ini adalah sebagai berikut.
\begin{itemize}
\item \texttt{public int sks()}\\ 
Jumlah bobot sks dari mata kuliah ini

\textbf{Parameter:}\begin{itemize}
\item Tidak memiliki parameter \textit{method}
\end{itemize}
\textbf{Return Value}: jumlah bobot sks

\textbf{Exception}: Tidak memiliki \textit{exception}

\item \texttt{public String nama()}\\ 
Nama mata kuliah ini

\textbf{Parameter:}\begin{itemize}
\item Tidak memiliki parameter \textit{method}
\end{itemize}
\textbf{Return Value}: nama mata kuliah

\textbf{Exception}: Tidak memiliki \textit{exception}

\end{itemize}
\item \texttt{MataKuliahFactory}\\ 
Kelas yang bertugas membuat kelas mata kuliah, dan menyimpannya untuk bisa
 digunakan kemudian (untuk hemat memori).

Atribut yang dimiliki kelas ini adalah sebagai berikut.
\begin{itemize}
\item \texttt{String DEFAULT\_MATAKULIAH\_PACKAGE} - Lokasi package untuk daftar mata kuliah
\item \texttt{MataKuliahFactory instance} - Singleton instance to factory.
\item \texttt{SortedMap mataKuliahCache} - Singleton instances untuk mata kuliah.
\end{itemize}
\textit{Method-method} yang dimiliki kelas ini adalah sebagai berikut.
\begin{itemize}
\item \texttt{public static MataKuliahFactory getInstance()}\\ 


\textbf{Parameter:}\begin{itemize}
\item Tidak memiliki parameter \textit{method}
\end{itemize}
\textbf{Return Value}: Tidak memiliki \textit{return value}

\textbf{Exception}: Tidak memiliki \textit{exception}

\item \texttt{public MataKuliah createMataKuliah(String kode, int sks, String nama)}\\ 
Membuat baru atau mendapatkan mata kuliah, jika memiliki informasi
 nama dan jumlah SKS.

\textbf{Parameter:}
\begin{itemize}
\item \texttt{String kode} - 
kode mata kuliah
\item \texttt{int sks} - 
jumlah SKS
\item \texttt{String nama} - 
nama mata kuliah
\end{itemize}
\textbf{Return Value}: objek mata kuliah

\textbf{Exception}: Tidak memiliki \textit{exception}

\item \texttt{public MataKuliah createMataKuliah(String kode)}\\ 
Membuat baru atau mendapatkan mata kuliah, jika tidak memiliki informasi
 nama dan jumlah SKS.

\textbf{Parameter:}
\begin{itemize}
\item \texttt{String kode} - 
kode mata kuliah
\end{itemize}
\textbf{Return Value}: objek mata kuliah

\textbf{Exception}: IllegalStateException
             jika sks dan tidak sesuai dengan yang ada di kode

\end{itemize}
\item \texttt{Semester}\\ 


Atribut yang dimiliki kelas ini adalah sebagai berikut.
\begin{itemize}
\item \texttt{Semester UNKNOWN5} - 
\item \texttt{Semester TRANSFER} - 
\item \texttt{Semester PENDEK} - 
\item \texttt{Semester GANJIL} - 
\item \texttt{Semester GENAP} - 
\item \texttt{int order} - 
\end{itemize}
\textit{Method-method} yang dimiliki kelas ini adalah sebagai berikut.
\begin{itemize}
\item \texttt{public static Semester values()}\\ 


\textbf{Parameter:}\begin{itemize}
\item Tidak memiliki parameter \textit{method}
\end{itemize}
\textbf{Return Value}: Tidak memiliki \textit{return value}

\textbf{Exception}: Tidak memiliki \textit{exception}

\item \texttt{public static Semester valueOf(String name)}\\ 


\textbf{Parameter:}\begin{itemize}
\item \texttt{String name} - 
\end{itemize}
\textbf{Return Value}: Tidak memiliki \textit{return value}

\textbf{Exception}: Tidak memiliki \textit{exception}

\item \texttt{public static Semester fromString(String text)}\\ 


\textbf{Parameter:}\begin{itemize}
\item \texttt{String text} - 
\end{itemize}
\textbf{Return Value}: Tidak memiliki \textit{return value}

\textbf{Exception}: Tidak memiliki \textit{exception}

\item \texttt{ int getOrder()}\\ 


\textbf{Parameter:}\begin{itemize}
\item Tidak memiliki parameter \textit{method}
\end{itemize}
\textbf{Return Value}: Tidak memiliki \textit{return value}

\textbf{Exception}: Tidak memiliki \textit{exception}

\end{itemize}
\item \texttt{TahunSemester}\\ 
Menyimpan konstanta untuk semester beserta tahunnya di UNPAR.

Atribut yang dimiliki kelas ini adalah sebagai berikut.
\begin{itemize}
\item \texttt{String kodeTahunSemester} - Kode semester 3 dijit, sesuai DPS:
 <ul>
   <li>2 dijit pertama berupa tahun, 2 dijit terakhir</li>
   <li>dijit terakhir: 1 untuk ganjil, 2 untuk genap, 4 untuk pendek.
 </ul>
\end{itemize}
\textit{Method-method} yang dimiliki kelas ini adalah sebagai berikut.
\begin{itemize}
\item \texttt{public Semester getSemester()}\\ 


\textbf{Parameter:}\begin{itemize}
\item Tidak memiliki parameter \textit{method}
\end{itemize}
\textbf{Return Value}: Tidak memiliki \textit{return value}

\textbf{Exception}: Tidak memiliki \textit{exception}

\item \texttt{public int getTahun()}\\ 


\textbf{Parameter:}\begin{itemize}
\item Tidak memiliki parameter \textit{method}
\end{itemize}
\textbf{Return Value}: Tidak memiliki \textit{return value}

\textbf{Exception}: Tidak memiliki \textit{exception}

\item \texttt{private static void validateKodeSemester(String kodeTahunSemester)}\\ 


\textbf{Parameter:}\begin{itemize}
\item \texttt{String kodeTahunSemester} - 
\end{itemize}
\textbf{Return Value}: Tidak memiliki \textit{return value}

\textbf{Exception}: Tidak memiliki \textit{exception}

\item \texttt{public String getKodeDPS()}\\ 
Mendapatkan kode tahun/semester sesuai aturan di DPS.

\textbf{Parameter:}\begin{itemize}
\item Tidak memiliki parameter \textit{method}
\end{itemize}
\textbf{Return Value}: kode tahun/semester sesuai aturan di DPS.

\textbf{Exception}: Tidak memiliki \textit{exception}

\item \texttt{public int compareTo(TahunSemester o)}\\ 


\textbf{Parameter:}\begin{itemize}
\item \texttt{TahunSemester o} - 
\end{itemize}
\textbf{Return Value}: Tidak memiliki \textit{return value}

\textbf{Exception}: Tidak memiliki \textit{exception}

\textbf{Override}: \texttt{compareTo} dari kelas \texttt{Object}

\item \texttt{public boolean equals(Object arg0)}\\ 


\textbf{Parameter:}\begin{itemize}
\item \texttt{Object arg0} - 
\end{itemize}
\textbf{Return Value}: Tidak memiliki \textit{return value}

\textbf{Exception}: Tidak memiliki \textit{exception}

\item \texttt{public String toString()}\\ 


\textbf{Parameter:}\begin{itemize}
\item Tidak memiliki parameter \textit{method}
\end{itemize}
\textbf{Return Value}: Tidak memiliki \textit{return value}

\textbf{Exception}: Tidak memiliki \textit{exception}

\end{itemize}
\item \texttt{Mahasiswa}\\ 


Atribut yang dimiliki kelas ini adalah sebagai berikut.
\begin{itemize}
\item \texttt{String npm} - 
\item \texttt{String nama} - 
\item \texttt{List riwayatNilai} - 
\item \texttt{URL photoURL} - 
\item \texttt{List jadwalKuliahList} - 
\item \texttt{SortedMap nilaiTOEFL} - 
\end{itemize}
\textit{Method-method} yang dimiliki kelas ini adalah sebagai berikut.
\begin{itemize}
\item \texttt{public String getNama()}\\ 


\textbf{Parameter:}\begin{itemize}
\item Tidak memiliki parameter \textit{method}
\end{itemize}
\textbf{Return Value}: Tidak memiliki \textit{return value}

\textbf{Exception}: Tidak memiliki \textit{exception}

\item \texttt{public void setNama(String nama)}\\ 


\textbf{Parameter:}\begin{itemize}
\item \texttt{String nama} - 
\end{itemize}
\textbf{Return Value}: Tidak memiliki \textit{return value}

\textbf{Exception}: Tidak memiliki \textit{exception}

\item \texttt{public String getNpm()}\\ 


\textbf{Parameter:}\begin{itemize}
\item Tidak memiliki parameter \textit{method}
\end{itemize}
\textbf{Return Value}: Tidak memiliki \textit{return value}

\textbf{Exception}: Tidak memiliki \textit{exception}

\item \texttt{public URL getPhotoURL()}\\ 


\textbf{Parameter:}\begin{itemize}
\item Tidak memiliki parameter \textit{method}
\end{itemize}
\textbf{Return Value}: Tidak memiliki \textit{return value}

\textbf{Exception}: Tidak memiliki \textit{exception}

\item \texttt{public void setPhotoURL(URL photoURL)}\\ 


\textbf{Parameter:}\begin{itemize}
\item \texttt{URL photoURL} - 
\end{itemize}
\textbf{Return Value}: Tidak memiliki \textit{return value}

\textbf{Exception}: Tidak memiliki \textit{exception}

\item \texttt{public List getJadwalKuliahList()}\\ 


\textbf{Parameter:}\begin{itemize}
\item Tidak memiliki parameter \textit{method}
\end{itemize}
\textbf{Return Value}: Tidak memiliki \textit{return value}

\textbf{Exception}: Tidak memiliki \textit{exception}

\item \texttt{public void setJadwalKuliahList(java.util.List jadwalKuliahList)}\\ 


\textbf{Parameter:}\begin{itemize}
\item \texttt{java.util.List jadwalKuliahList} - 
\end{itemize}
\textbf{Return Value}: Tidak memiliki \textit{return value}

\textbf{Exception}: Tidak memiliki \textit{exception}

\item \texttt{public String getEmailAddress()}\\ 


\textbf{Parameter:}\begin{itemize}
\item Tidak memiliki parameter \textit{method}
\end{itemize}
\textbf{Return Value}: Tidak memiliki \textit{return value}

\textbf{Exception}: Tidak memiliki \textit{exception}

\item \texttt{public List getRiwayatNilai()}\\ 


\textbf{Parameter:}\begin{itemize}
\item Tidak memiliki parameter \textit{method}
\end{itemize}
\textbf{Return Value}: Tidak memiliki \textit{return value}

\textbf{Exception}: Tidak memiliki \textit{exception}

\item \texttt{public SortedMap getNilaiTOEFL()}\\ 


\textbf{Parameter:}\begin{itemize}
\item Tidak memiliki parameter \textit{method}
\end{itemize}
\textbf{Return Value}: Tidak memiliki \textit{return value}

\textbf{Exception}: Tidak memiliki \textit{exception}

\item \texttt{public void setNilaiTOEFL(java.util.SortedMap nilaiTOEFL)}\\ 


\textbf{Parameter:}\begin{itemize}
\item \texttt{java.util.SortedMap nilaiTOEFL} - 
\end{itemize}
\textbf{Return Value}: Tidak memiliki \textit{return value}

\textbf{Exception}: Tidak memiliki \textit{exception}

\item \texttt{public double calculateIPKLulus()}\\ 
Menghitung IPK mahasiswa sampai saat ini, dengan aturan:
 <ul>
   <li>Kuliah yang tidak lulus tidak dihitung
   <li>Jika pengambilan beberapa kali, diambil <em>nilai terbaik</em>.
 </ul>
 Sebelum memanggil method ini, \texttt{getRiwayatNilai()} harus sudah mengandung nilai per mata kuliah!

\textbf{Parameter:}\begin{itemize}
\item Tidak memiliki parameter \textit{method}
\end{itemize}
\textbf{Return Value}: IPK Lulus, atau  DoubleNaN jika belum mengambil satu kuliah pun.

\textbf{Exception}: Tidak memiliki \textit{exception}

\item \texttt{public double calculateIPLulus()}\\ 
Menghitung IP mahasiswa sampai saat ini, dengan aturan:
 <ul>
   <li>Kuliah yang tidak lulus tidak dihitung
   <li>Jika pengambilan beberapa kali, diambil <em>nilai terbaik</em>.
 </ul>
 Sebelum memanggil method ini, \texttt{getRiwayatNilai()} harus sudah mengandung nilai per mata kuliah!

\textbf{Parameter:}\begin{itemize}
\item Tidak memiliki parameter \textit{method}
\end{itemize}
\textbf{Return Value}: IPK Lulus, atau  DoubleNaN jika belum mengambil satu kuliah pun.

\textbf{Exception}: Tidak memiliki \textit{exception}

\item \texttt{public double calculateIPTempuh(boolean lulusSaja)}\\ 
Menghitung IP mahasiswa sampai saat ini, dengan aturan:
 <ul>
   <li>Perhitungan kuliah yang tidak lulus ditentukan parameter
   <li>Jika pengambilan beberapa kali, diambil <em>nilai terbaik</em>.
 </ul>

\textbf{Parameter:}
\begin{itemize}
\item \texttt{boolean lulusSaja} - 
set true jika ingin membuang mata kuliah tidak lulus, false jika ingin semua (sama dengan "IP N. Terbaik" di DPS)
 Sebelum memanggil method ini, \texttt{getRiwayatNilai()} harus sudah mengandung nilai per mata kuliah!
\end{itemize}
\textbf{Return Value}: IPK Lulus, atau  DoubleNaN jika belum mengambil satu kuliah pun.

\textbf{Exception}: Tidak memiliki \textit{exception}

\item \texttt{public double calculateIPKumulatif()}\\ 
Menghitung IP Kumulatif mahasiswa sampai saat ini, dengan aturan:
 <ul>
   <li>Jika pengambilan beberapa kali, diambil semua.
 </ul>
 Sebelum memanggil method ini, \texttt{getRiwayatNilai()} harus sudah mengandung nilai per mata kuliah!

\textbf{Parameter:}\begin{itemize}
\item Tidak memiliki parameter \textit{method}
\end{itemize}
\textbf{Return Value}: IPK Lulus, atau  DoubleNaN jika belum mengambil satu kuliah pun.

\textbf{Exception}: Tidak memiliki \textit{exception}

\item \texttt{public double calculateIPKTempuh(boolean lulusSaja)}\\ 
Menghitung IPK mahasiswa sampai saat ini, dengan aturan:
 <ul>
   <li>Perhitungan kuliah yang tidak lulus ditentukan parameter
   <li>Jika pengambilan beberapa kali, diambil <em>nilai terbaik</em>.
 </ul>

\textbf{Parameter:}
\begin{itemize}
\item \texttt{boolean lulusSaja} - 
set true jika ingin membuang mata kuliah tidak lulus 
 Sebelum memanggil method ini, \texttt{getRiwayatNilai()} harus sudah mengandung nilai per mata kuliah!
\end{itemize}
\textbf{Return Value}: IPK Lulus, atau  DoubleNaN jika belum mengambil satu kuliah pun.

\textbf{Exception}: Tidak memiliki \textit{exception}

\item \texttt{public double calculateIPS()}\\ 
Menghitung IPS semester terakhir sampai saat ini, dengan aturan:
 <ul>
   <li>Kuliah yang tidak lulus <em>dihitung</em>.
 </ul>
 Sebelum memanggil method ini, \texttt{getRiwayatNilai()} harus sudah mengandung nilai per mata kuliah!

\textbf{Parameter:}\begin{itemize}
\item Tidak memiliki parameter \textit{method}
\end{itemize}
\textbf{Return Value}: nilai IPS sampai saat ini

\textbf{Exception}: ArrayIndexOutOfBoundsException jika belum ada nilai satupun

\item \texttt{public int calculateSKSLulus()}\\ 
Menghitung jumlah SKS lulus mahasiswa saat ini.
 Sebelum memanggil method ini, \texttt{getRiwayatNilai()} harus sudah mengandung nilai per mata kuliah!

\textbf{Parameter:}\begin{itemize}
\item Tidak memiliki parameter \textit{method}
\end{itemize}
\textbf{Return Value}: SKS Lulus

\textbf{Exception}: Tidak memiliki \textit{exception}

\item \texttt{public int calculateSKSTempuh(boolean lulusSaja)}\\ 
Menghitung jumlah SKS tempuh mahasiswa saat ini.
 Sebelum memanggil method ini, \texttt{getRiwayatNilai()} harus sudah mengandung nilai per mata kuliah!

\textbf{Parameter:}
\begin{itemize}
\item \texttt{boolean lulusSaja} - 
set true jika ingin membuang SKS tidak lulus
\end{itemize}
\textbf{Return Value}: SKS tempuh

\textbf{Exception}: Tidak memiliki \textit{exception}

\item \texttt{public Set calculateTahunSemesterAktif()}\\ 
Mendapatkan seluruh tahun semester di mana mahasiswa ini tercatat
 sebagai mahasiswa aktif, dengan strategi memeriksa riwayat nilainya.
 Jika ada satu nilai saja pada sebuah tahun semester, maka dianggap
 aktif pada semester tersebut.

\textbf{Parameter:}\begin{itemize}
\item Tidak memiliki parameter \textit{method}
\end{itemize}
\textbf{Return Value}: kumpulan tahun semester di mana mahasiswa ini aktif

\textbf{Exception}: Tidak memiliki \textit{exception}

\item \texttt{public boolean hasLulusKuliah(String kodeMataKuliah)}\\ 
Memeriksa apakah mahasiswa ini sudah lulus mata kuliah tertentu. Kompleksitas O(n).
 Sebelum memanggil method ini, \texttt{getRiwayatNilai()} harus sudah mengandung nilai per mata kuliah!
 Note: jika yang dimiliki adalah MataKuliah, gunakanlah MataKuliahgetKode().

\textbf{Parameter:}
\begin{itemize}
\item \texttt{String kodeMataKuliah} - 
kode mata kuliah yang ingin diperiksa kelulusannya.
\end{itemize}
\textbf{Return Value}: true jika sudah pernah mengambil dan lulus, false jika belum

\textbf{Exception}: Tidak memiliki \textit{exception}

\item \texttt{public boolean hasTempuhKuliah(String kodeMataKuliah)}\\ 
Memeriksa apakah mahasiswa ini sudah pernah menempuh mata kuliah tertentu. Kompleksitas O(n).
 Sebelum memanggil method ini, \texttt{getRiwayatNilai()} harus sudah ada isinya!
 Note: jika yang dimiliki adalah MataKuliah, gunakanlah MataKuliahgetKode().

\textbf{Parameter:}
\begin{itemize}
\item \texttt{String kodeMataKuliah} - 
kode mata kuliah yang ingin diperiksa.
\end{itemize}
\textbf{Return Value}: true jika sudah pernah mengambil, false jika belum

\textbf{Exception}: Tidak memiliki \textit{exception}

\item \texttt{public int getTahunAngkatan()}\\ 
Mendapatkan tahun angkatan mahasiswa ini, berdasarkan NPM nya

\textbf{Parameter:}\begin{itemize}
\item Tidak memiliki parameter \textit{method}
\end{itemize}
\textbf{Return Value}: tahun angkatan

\textbf{Exception}: Tidak memiliki \textit{exception}

\item \texttt{public String toString()}\\ 


\textbf{Parameter:}\begin{itemize}
\item Tidak memiliki parameter \textit{method}
\end{itemize}
\textbf{Return Value}: Tidak memiliki \textit{return value}

\textbf{Exception}: Tidak memiliki \textit{exception}

\end{itemize}
\item \texttt{Mahasiswa.Nilai}\\ 
Merepresentasikan nilai yang ada di riwayat nilai mahasiswa

Atribut yang dimiliki kelas ini adalah sebagai berikut.
\begin{itemize}
\item \texttt{TahunSemester tahunSemester} - Tahun dan Semester kuliah ini diambil
\item \texttt{MataKuliah mataKuliah} - Mata kuliah yang diambil
\item \texttt{Character kelas} - Kelas kuliah
\item \texttt{Double nilaiART} - Nilai ART
\item \texttt{Double nilaiUTS} - Nilai UTS
\item \texttt{Double nilaiUAS} - Nilai UAS
\item \texttt{Character nilaiAkhir} - Nilai Akhir
\end{itemize}
\textit{Method-method} yang dimiliki kelas ini adalah sebagai berikut.
\begin{itemize}
\item \texttt{public MataKuliah getMataKuliah()}\\ 


\textbf{Parameter:}\begin{itemize}
\item Tidak memiliki parameter \textit{method}
\end{itemize}
\textbf{Return Value}: Tidak memiliki \textit{return value}

\textbf{Exception}: Tidak memiliki \textit{exception}

\item \texttt{public Character getKelas()}\\ 


\textbf{Parameter:}\begin{itemize}
\item Tidak memiliki parameter \textit{method}
\end{itemize}
\textbf{Return Value}: Tidak memiliki \textit{return value}

\textbf{Exception}: Tidak memiliki \textit{exception}

\item \texttt{public Double getNilaiART()}\\ 


\textbf{Parameter:}\begin{itemize}
\item Tidak memiliki parameter \textit{method}
\end{itemize}
\textbf{Return Value}: Tidak memiliki \textit{return value}

\textbf{Exception}: Tidak memiliki \textit{exception}

\item \texttt{public Double getNilaiUTS()}\\ 


\textbf{Parameter:}\begin{itemize}
\item Tidak memiliki parameter \textit{method}
\end{itemize}
\textbf{Return Value}: Tidak memiliki \textit{return value}

\textbf{Exception}: Tidak memiliki \textit{exception}

\item \texttt{public Double getNilaiUAS()}\\ 


\textbf{Parameter:}\begin{itemize}
\item Tidak memiliki parameter \textit{method}
\end{itemize}
\textbf{Return Value}: Tidak memiliki \textit{return value}

\textbf{Exception}: Tidak memiliki \textit{exception}

\item \texttt{public Character getNilaiAkhir()}\\ 
Mengembalikan nilai akhir dalam bentuk huruf (A, B, C, D, ..., atau K)

\textbf{Parameter:}\begin{itemize}
\item Tidak memiliki parameter \textit{method}
\end{itemize}
\textbf{Return Value}: nilai akhir dalam huruf, atau null jika tidak ada.

\textbf{Exception}: Tidak memiliki \textit{exception}

\item \texttt{public Double getAngkaAkhir()}\\ 
Mendapatkan nilai akhir dalam bentuk angka

\textbf{Parameter:}\begin{itemize}
\item Tidak memiliki parameter \textit{method}
\end{itemize}
\textbf{Return Value}: nilai akhir dalam angka, atau null jika getNilaiAkhir() mengembalikan 'K' atau null

\textbf{Exception}: Tidak memiliki \textit{exception}

\item \texttt{public TahunSemester getTahunSemester()}\\ 


\textbf{Parameter:}\begin{itemize}
\item Tidak memiliki parameter \textit{method}
\end{itemize}
\textbf{Return Value}: Tidak memiliki \textit{return value}

\textbf{Exception}: Tidak memiliki \textit{exception}

\item \texttt{public int getTahunAjaran()}\\ 


\textbf{Parameter:}\begin{itemize}
\item Tidak memiliki parameter \textit{method}
\end{itemize}
\textbf{Return Value}: Tidak memiliki \textit{return value}

\textbf{Exception}: Tidak memiliki \textit{exception}

\item \texttt{public Semester getSemester()}\\ 


\textbf{Parameter:}\begin{itemize}
\item Tidak memiliki parameter \textit{method}
\end{itemize}
\textbf{Return Value}: Tidak memiliki \textit{return value}

\textbf{Exception}: Tidak memiliki \textit{exception}

\item \texttt{public String toString()}\\ 


\textbf{Parameter:}\begin{itemize}
\item Tidak memiliki parameter \textit{method}
\end{itemize}
\textbf{Return Value}: Tidak memiliki \textit{return value}

\textbf{Exception}: Tidak memiliki \textit{exception}

\end{itemize}
\item \texttt{Mahasiswa.Nilai.ChronologicalComparator}\\ 
Pembanding antara satu nilai dengan nilai lainnya, secara
 kronologis waktu pengambilan.

Kelas ini tidak memiliki atribut. \textit{Method-method} yang dimiliki kelas ini adalah sebagai berikut.
\begin{itemize}
\item \texttt{public int compare(Mahasiswa.Nilai o1, Mahasiswa.Nilai o2)}\\ 


\textbf{Parameter:}\begin{itemize}
\item \texttt{Mahasiswa.Nilai o1} - 
\item \texttt{Mahasiswa.Nilai o2} - 
\end{itemize}
\textbf{Return Value}: Tidak memiliki \textit{return value}

\textbf{Exception}: Tidak memiliki \textit{exception}

\textbf{Override}: \texttt{compare} dari kelas \texttt{Object}

\end{itemize}
\item \texttt{JadwalKuliah}\\ 


Atribut yang dimiliki kelas ini adalah sebagai berikut.
\begin{itemize}
\item \texttt{MataKuliah mataKuliah} - 
\item \texttt{Character kelas} - 
\item \texttt{DayOfWeek hari} - 
\item \texttt{LocalTime waktuMulai} - 
\item \texttt{LocalTime waktuSelesai} - 
\item \texttt{String lokasi} - 
\item \texttt{Dosen pengajar} - 
\end{itemize}
\textit{Method-method} yang dimiliki kelas ini adalah sebagai berikut.
\begin{itemize}
\item \texttt{public MataKuliah getMataKuliah()}\\ 


\textbf{Parameter:}\begin{itemize}
\item Tidak memiliki parameter \textit{method}
\end{itemize}
\textbf{Return Value}: Tidak memiliki \textit{return value}

\textbf{Exception}: Tidak memiliki \textit{exception}

\item \texttt{public void setMataKuliah(MataKuliah mataKuliah)}\\ 


\textbf{Parameter:}\begin{itemize}
\item \texttt{MataKuliah mataKuliah} - 
\end{itemize}
\textbf{Return Value}: Tidak memiliki \textit{return value}

\textbf{Exception}: Tidak memiliki \textit{exception}

\item \texttt{public Character getKelas()}\\ 


\textbf{Parameter:}\begin{itemize}
\item Tidak memiliki parameter \textit{method}
\end{itemize}
\textbf{Return Value}: Tidak memiliki \textit{return value}

\textbf{Exception}: Tidak memiliki \textit{exception}

\item \texttt{public void setKelas(Character kelas)}\\ 


\textbf{Parameter:}\begin{itemize}
\item \texttt{Character kelas} - 
\end{itemize}
\textbf{Return Value}: Tidak memiliki \textit{return value}

\textbf{Exception}: Tidak memiliki \textit{exception}

\item \texttt{public DayOfWeek getHari()}\\ 


\textbf{Parameter:}\begin{itemize}
\item Tidak memiliki parameter \textit{method}
\end{itemize}
\textbf{Return Value}: Tidak memiliki \textit{return value}

\textbf{Exception}: Tidak memiliki \textit{exception}

\item \texttt{public void setHari(DayOfWeek hari)}\\ 


\textbf{Parameter:}\begin{itemize}
\item \texttt{DayOfWeek hari} - 
\end{itemize}
\textbf{Return Value}: Tidak memiliki \textit{return value}

\textbf{Exception}: Tidak memiliki \textit{exception}

\item \texttt{public LocalTime getWaktuMulai()}\\ 


\textbf{Parameter:}\begin{itemize}
\item Tidak memiliki parameter \textit{method}
\end{itemize}
\textbf{Return Value}: Tidak memiliki \textit{return value}

\textbf{Exception}: Tidak memiliki \textit{exception}

\item \texttt{public void setWaktuMulai(LocalTime waktuMulai)}\\ 


\textbf{Parameter:}\begin{itemize}
\item \texttt{LocalTime waktuMulai} - 
\end{itemize}
\textbf{Return Value}: Tidak memiliki \textit{return value}

\textbf{Exception}: Tidak memiliki \textit{exception}

\item \texttt{public LocalTime getWaktuSelesai()}\\ 


\textbf{Parameter:}\begin{itemize}
\item Tidak memiliki parameter \textit{method}
\end{itemize}
\textbf{Return Value}: Tidak memiliki \textit{return value}

\textbf{Exception}: Tidak memiliki \textit{exception}

\item \texttt{public void setWaktuSelesai(LocalTime waktuSelesai)}\\ 


\textbf{Parameter:}\begin{itemize}
\item \texttt{LocalTime waktuSelesai} - 
\end{itemize}
\textbf{Return Value}: Tidak memiliki \textit{return value}

\textbf{Exception}: Tidak memiliki \textit{exception}

\item \texttt{public String getLokasi()}\\ 


\textbf{Parameter:}\begin{itemize}
\item Tidak memiliki parameter \textit{method}
\end{itemize}
\textbf{Return Value}: Tidak memiliki \textit{return value}

\textbf{Exception}: Tidak memiliki \textit{exception}

\item \texttt{public void setLokasi(String lokasi)}\\ 


\textbf{Parameter:}\begin{itemize}
\item \texttt{String lokasi} - 
\end{itemize}
\textbf{Return Value}: Tidak memiliki \textit{return value}

\textbf{Exception}: Tidak memiliki \textit{exception}

\item \texttt{public Dosen getPengajar()}\\ 


\textbf{Parameter:}\begin{itemize}
\item Tidak memiliki parameter \textit{method}
\end{itemize}
\textbf{Return Value}: Tidak memiliki \textit{return value}

\textbf{Exception}: Tidak memiliki \textit{exception}

\item \texttt{public void setPengajar(Dosen pengajar)}\\ 


\textbf{Parameter:}\begin{itemize}
\item \texttt{Dosen pengajar} - 
\end{itemize}
\textbf{Return Value}: Tidak memiliki \textit{return value}

\textbf{Exception}: Tidak memiliki \textit{exception}

\item \texttt{public String getWaktuString()}\\ 


\textbf{Parameter:}\begin{itemize}
\item Tidak memiliki parameter \textit{method}
\end{itemize}
\textbf{Return Value}: Tidak memiliki \textit{return value}

\textbf{Exception}: Tidak memiliki \textit{exception}

\item \texttt{public static DayOfWeek indonesianToDayOfWeek(String indonesian)}\\ 
Converts Indonesian day names to \texttt{DayOfWeek} enumeration.

\textbf{Parameter:}
\begin{itemize}
\item \texttt{String indonesian} - 
the day name in Indonesian
\end{itemize}
\textbf{Return Value}: DayOfWeek object or null if not found.

\textbf{Exception}: Tidak memiliki \textit{exception}

\end{itemize}
\item \texttt{MataKuliah}\\ 


Atribut yang dimiliki kelas ini adalah sebagai berikut.
\begin{itemize}
\item \texttt{String kode} - 
\item \texttt{String nama} - 
\item \texttt{Integer sks} - 
\end{itemize}
\textit{Method-method} yang dimiliki kelas ini adalah sebagai berikut.
\begin{itemize}
\item \texttt{public String getKode()}\\ 


\textbf{Parameter:}\begin{itemize}
\item Tidak memiliki parameter \textit{method}
\end{itemize}
\textbf{Return Value}: Tidak memiliki \textit{return value}

\textbf{Exception}: Tidak memiliki \textit{exception}

\item \texttt{public String getNama()}\\ 


\textbf{Parameter:}\begin{itemize}
\item Tidak memiliki parameter \textit{method}
\end{itemize}
\textbf{Return Value}: Tidak memiliki \textit{return value}

\textbf{Exception}: Tidak memiliki \textit{exception}

\item \texttt{public Integer getSks()}\\ 


\textbf{Parameter:}\begin{itemize}
\item Tidak memiliki parameter \textit{method}
\end{itemize}
\textbf{Return Value}: Tidak memiliki \textit{return value}

\textbf{Exception}: Tidak memiliki \textit{exception}

\end{itemize}
\item \texttt{Dosen}\\ 


Atribut yang dimiliki kelas ini adalah sebagai berikut.
\begin{itemize}
\item \texttt{String nik} - 
\item \texttt{String nama} - 
\end{itemize}
\textit{Method-method} yang dimiliki kelas ini adalah sebagai berikut.
\begin{itemize}
\item \texttt{public String getNik()}\\ 


\textbf{Parameter:}\begin{itemize}
\item Tidak memiliki parameter \textit{method}
\end{itemize}
\textbf{Return Value}: Tidak memiliki \textit{return value}

\textbf{Exception}: Tidak memiliki \textit{exception}

\item \texttt{public void setNik(String nik)}\\ 


\textbf{Parameter:}\begin{itemize}
\item \texttt{String nik} - 
\end{itemize}
\textbf{Return Value}: Tidak memiliki \textit{return value}

\textbf{Exception}: Tidak memiliki \textit{exception}

\item \texttt{public String getNama()}\\ 


\textbf{Parameter:}\begin{itemize}
\item Tidak memiliki parameter \textit{method}
\end{itemize}
\textbf{Return Value}: Tidak memiliki \textit{return value}

\textbf{Exception}: Tidak memiliki \textit{exception}

\item \texttt{public void setNama(String nama)}\\ 


\textbf{Parameter:}\begin{itemize}
\item \texttt{String nama} - 
\end{itemize}
\textbf{Return Value}: Tidak memiliki \textit{return value}

\textbf{Exception}: Tidak memiliki \textit{exception}

\item \texttt{public boolean equals(Object arg0)}\\ 


\textbf{Parameter:}\begin{itemize}
\item \texttt{Object arg0} - 
\end{itemize}
\textbf{Return Value}: Tidak memiliki \textit{return value}

\textbf{Exception}: Tidak memiliki \textit{exception}

\end{itemize}
\item \texttt{MKU008}\\ 
Mendalami perilaku sehari-hari yang baik dalam bermasyarakat.

Kelas ini tidak memiliki atribut. Kelas ini tidak memiliki method. \item \texttt{IIE210}\\ 


Kelas ini tidak memiliki atribut. Kelas ini tidak memiliki method. \item \texttt{AIF203}\\ 
Mata kuliah ini memperkenalkan kepada mahasiswa konsep struktur diskret yang 
 digunakan pada bidang informatika diantaranya graph, pohon dan finite state 
 machine

Kelas ini tidak memiliki atribut. \textit{Method-method} yang dimiliki kelas ini adalah sebagai berikut.
\begin{itemize}
\item \texttt{public boolean checkPrasyarat(Mahasiswa mahasiswa, java.util.List reasonsContainer)}\\ 


\textbf{Parameter:}\begin{itemize}
\item \texttt{Mahasiswa mahasiswa} - 
\item \texttt{java.util.List reasonsContainer} - 
\end{itemize}
\textbf{Return Value}: Tidak memiliki \textit{return value}

\textbf{Exception}: Tidak memiliki \textit{exception}

\textbf{Override}: \texttt{checkPrasyarat} dari kelas \texttt{MataKuliah}

\end{itemize}
\item \texttt{AIF311}\\ 
Kuliah Pemrograman Fungsional bertujuan untuk: 1. memperkenalkan paradigma
 pemrograman fungsional, yaitu sebuah pemrograman yang didasarkan pada konsep
 pemetaan dan fungsi matematika. Penyelesaian suatu masalah didasari atas
 aplikasi dari fungsi-fungsi tersebut. 2. memberikan dasar-dasar pemrograman
 fungsional dengan menggunakan bahasa fungsional Haskell.

Kelas ini tidak memiliki atribut. \textit{Method-method} yang dimiliki kelas ini adalah sebagai berikut.
\begin{itemize}
\item \texttt{public boolean checkPrasyarat(Mahasiswa mahasiswa, java.util.List reasonsContainer)}\\ 


\textbf{Parameter:}\begin{itemize}
\item \texttt{Mahasiswa mahasiswa} - 
\item \texttt{java.util.List reasonsContainer} - 
\end{itemize}
\textbf{Return Value}: Tidak memiliki \textit{return value}

\textbf{Exception}: Tidak memiliki \textit{exception}

\textbf{Override}: \texttt{checkPrasyarat} dari kelas \texttt{MataKuliah}

\end{itemize}
\item \texttt{AIF192}\\ 


Kelas ini tidak memiliki atribut. Kelas ini tidak memiliki method. \item \texttt{AIF468}\\ 


Kelas ini tidak memiliki atribut. Kelas ini tidak memiliki method. \item \texttt{IIE103}\\ 


Kelas ini tidak memiliki atribut. Kelas ini tidak memiliki method. \item \texttt{AIF385}\\ 


Kelas ini tidak memiliki atribut. Kelas ini tidak memiliki method. \item \texttt{AIF106}\\ 
Mata kuliah ini memberikan pengetahuan tentang cara kerja komputer, dimulai 
 dari representasi data dan berbagai macam operasinya. Selanjutnya, juga 
 diperkenalkan bagaimana merepresentasikan suatu fungsi dalam rangkaian 
 gerbang logika, dan bagaimana menyederhanakannya. Berbagai rangkaian dasar 
 yang digunakan di dalam komputer juga dipekenalkan. Mahasiswa juga akan 
 mempelajari komponen komputer, misalnya register dan memori.

Kelas ini tidak memiliki atribut. Kelas ini tidak memiliki method. \item \texttt{AIF281}\\ 


Kelas ini tidak memiliki atribut. Kelas ini tidak memiliki method. \item \texttt{EAA102}\\ 


Kelas ini tidak memiliki atribut. Kelas ini tidak memiliki method. \item \texttt{AIF405}\\ 
Mata kuliah ini merupakan lanjutan dari Projek Sistem Informasi 1 dan
 memberikan kesempatan bagi mahasiswa untuk melanjutkan/mengembangkan
 perancangan sitem pada organisasi studi kasus, mengimplementasikan rancangan
 dan melakukan pengujian perangkat lunak;

Kelas ini tidak memiliki atribut. \textit{Method-method} yang dimiliki kelas ini adalah sebagai berikut.
\begin{itemize}
\item \texttt{public boolean checkPrasyarat(Mahasiswa mahasiswa, java.util.List reasonsContainer)}\\ 


\textbf{Parameter:}\begin{itemize}
\item \texttt{Mahasiswa mahasiswa} - 
\item \texttt{java.util.List reasonsContainer} - 
\end{itemize}
\textbf{Return Value}: Tidak memiliki \textit{return value}

\textbf{Exception}: Tidak memiliki \textit{exception}

\textbf{Override}: \texttt{checkPrasyarat} dari kelas \texttt{MataKuliah}

\end{itemize}
\item \texttt{APS182}\\ 


Kelas ini tidak memiliki atribut. Kelas ini tidak memiliki method. \item \texttt{MKU004}\\ 
Fenomenologi Agama merupakan bagian yang tak terpisahkan dari kajian filosofis, kritis, 
 rasional, dan obyektif mengenai substansi ajaran agama. Fenomenologi merupakan sebuah 
 disiplin ilmu yang secara kritis-rasional mengkaji fenomena dan dinamika kehidupan manusia 
 beragama, dari upaya menjadikan Tuhan sebagai tujuan sesembahan sampai menempatkan Tuhan 
 sebagai instrumen legitimasi untuk melakukan tindakan yang justru bertolak belakang dengan 
 kehendak Tuhan yang disembah. Sehubungan dengan itu, kritik konstruktif terhadap perilaku 
 manusia beragama menjadi salah satu poin utama dalam mata kuliah ini. Kesediaan untuk 
 melakukan otoritik terhadap agama sendiri erat terkait dengan upaya menemukan kembali nilai
 sejati agama atau otentisitas hidup beragama.

Kelas ini tidak memiliki atribut. Kelas ini tidak memiliki method. \item \texttt{MKU012}\\ 
Perkuliahan logika ditujukan untuk memberikan dasar-dasar ketrampilan berpikir rasional dan
 sistematik. Isinya mencakup ketrampilan berpikir deduktif dan induktif, seperti silogisme, 
 argumen analogikal dan generalisasi induktif. Pembahasan teoretis disertai pula dengan
 pelatihan praktis yang diarahkan pada proses berpikir. Untuk menajamkan kemampuan berpikir 
 tersebut, mahasiswa dilatih pula mengidentifikasi kerancuan-kerancuan (fallacies) yang sering 
 dijumpai baik dalam kehidupan sehari-hari maupun dalam konteks akademik.

Kelas ini tidak memiliki atribut. Kelas ini tidak memiliki method. \item \texttt{AIF389}\\ 


Kelas ini tidak memiliki atribut. Kelas ini tidak memiliki method. \item \texttt{AMS190}\\ 


Kelas ini tidak memiliki atribut. Kelas ini tidak memiliki method. \item \texttt{AMS191}\\ 


Kelas ini tidak memiliki atribut. Kelas ini tidak memiliki method. \item \texttt{AMS200}\\ 


Kelas ini tidak memiliki atribut. Kelas ini tidak memiliki method. \item \texttt{AIF330}\\ 


Kelas ini tidak memiliki atribut. Kelas ini tidak memiliki method. \item \texttt{AIF388}\\ 


Kelas ini tidak memiliki atribut. Kelas ini tidak memiliki method. \item \texttt{AIF465}\\ 


Kelas ini tidak memiliki atribut. Kelas ini tidak memiliki method. \item \texttt{AIF453}\\ 
Mata kuliah inimemperkenalkan kebutuhan organisasi terhadap sistem business
 intelligent (BI) dan pemanfaatan BI untuk organisasi; memperkenalkan konsep
 sistem business intelligent dan komponennya; Mempelajari tenik-teknik
 analisis data bisnis dan visualisasi hasil analisis; Mempelajari konsep data
 warehouse dan perancangannya dan fungsi OLAP; Mempraktekkan teknik-teknik
 analisis data dan visualisasi hasil analisis.

Kelas ini tidak memiliki atribut. \textit{Method-method} yang dimiliki kelas ini adalah sebagai berikut.
\begin{itemize}
\item \texttt{public boolean checkPrasyarat(Mahasiswa mahasiswa, java.util.List reasonsContainer)}\\ 


\textbf{Parameter:}\begin{itemize}
\item \texttt{Mahasiswa mahasiswa} - 
\item \texttt{java.util.List reasonsContainer} - 
\end{itemize}
\textbf{Return Value}: Tidak memiliki \textit{return value}

\textbf{Exception}: Tidak memiliki \textit{exception}

\textbf{Override}: \texttt{checkPrasyarat} dari kelas \texttt{MataKuliah}

\end{itemize}
\item \texttt{AIF280}\\ 


Kelas ini tidak memiliki atribut. Kelas ini tidak memiliki method. \item \texttt{AIF445}\\ 


Kelas ini tidak memiliki atribut. \textit{Method-method} yang dimiliki kelas ini adalah sebagai berikut.
\begin{itemize}
\item \texttt{public boolean checkPrasyarat(Mahasiswa mahasiswa, java.util.List reasonsContainer)}\\ 


\textbf{Parameter:}\begin{itemize}
\item \texttt{Mahasiswa mahasiswa} - 
\item \texttt{java.util.List reasonsContainer} - 
\end{itemize}
\textbf{Return Value}: Tidak memiliki \textit{return value}

\textbf{Exception}: Tidak memiliki \textit{exception}

\textbf{Override}: \texttt{checkPrasyarat} dari kelas \texttt{MataKuliah}

\end{itemize}
\item \texttt{AIF469}\\ 
Mata kuliah ini mengajarkan kepada mahasiswa teknik-teknik untuk membuat 
 layanan berbasis web. Mahasiswa diperkenalkan dengan standar-standar seperti 
 HTTP, XML, JSON dan diajarkan untuk memanfaatkannya dalam membuat maupun 
 menggunakan layanan pihak ketiga. Dalam kuliah ini, juga akan diperkenalkan 
 minimal satu layanan pihak ketiga yang dapat dimanfaatkan mahasiswa, seperti 
 Google Places Web Service.

Kelas ini tidak memiliki atribut. \textit{Method-method} yang dimiliki kelas ini adalah sebagai berikut.
\begin{itemize}
\item \texttt{public boolean checkPrasyarat(Mahasiswa mahasiswa, java.util.List reasonsContainer)}\\ 


\textbf{Parameter:}\begin{itemize}
\item \texttt{Mahasiswa mahasiswa} - 
\item \texttt{java.util.List reasonsContainer} - 
\end{itemize}
\textbf{Return Value}: Tidak memiliki \textit{return value}

\textbf{Exception}: Tidak memiliki \textit{exception}

\textbf{Override}: \texttt{checkPrasyarat} dari kelas \texttt{MataKuliah}

\end{itemize}
\item \texttt{AIF486}\\ 


Kelas ini tidak memiliki atribut. Kelas ini tidak memiliki method. \item \texttt{IIE207}\\ 


Kelas ini tidak memiliki atribut. Kelas ini tidak memiliki method. \item \texttt{AIF202}\\ 
Mata kuliah ini memperkenalkan kepada mahasiswa beberapa algoritma dan 
 struktur data, alternatif cara implementasinya, dan analisis kompleksitas 
 waktunya. Mahasiswa diberikan beberapa masalah komputasi yang harus 
 diselesaikan dengan menggunakan algoritma atau struktur data yang sudah 
 diperkenalkan dan mengimplementasikannya dalam bahasa pemrograman Java.

Kelas ini tidak memiliki atribut. \textit{Method-method} yang dimiliki kelas ini adalah sebagai berikut.
\begin{itemize}
\item \texttt{public boolean checkPrasyarat(Mahasiswa mahasiswa, java.util.List reasonsContainer)}\\ 


\textbf{Parameter:}\begin{itemize}
\item \texttt{Mahasiswa mahasiswa} - 
\item \texttt{java.util.List reasonsContainer} - 
\end{itemize}
\textbf{Return Value}: Tidak memiliki \textit{return value}

\textbf{Exception}: Tidak memiliki \textit{exception}

\textbf{Override}: \texttt{checkPrasyarat} dari kelas \texttt{MataKuliah}

\end{itemize}
\item \texttt{AIF347}\\ 


Kelas ini tidak memiliki atribut. Kelas ini tidak memiliki method. \item \texttt{MKU009}\\ 
Tujuan dari mata kuliah ini adalah untuk mendalami keterampilan berbahasa Indonesia, agar 
 mampu mengkomunikasikan hasil pemikiran serta meningkatkan keterampilan dalam menyusun karya
 ilmiah. Mata kuliah Bahasa Indonesia ini dimulai dengan mempelajari penulisan kata baku dan 
 non baku serta pengungkapan pikiran dengan pungtuasi yang benar. Selanjutnya dipelajari 
 penyusunan kalimat yang baku serta menghubungkan kalimat-kalimat yang padu dalam menuangkan 
 gagasan dalam sebuah paragraf. Selain itu, dalam matakuliah ini dipelajari cara menyusun
 surat dinas yang jelas dan komunikatif. Di akhir kuliah ini, mahasiswa diberi tugas
 penyusunan makalah dengan benar.

Kelas ini tidak memiliki atribut. Kelas ini tidak memiliki method. \item \texttt{ESM203}\\ 


Kelas ini tidak memiliki atribut. Kelas ini tidak memiliki method. \item \texttt{APS402}\\ 


Kelas ini tidak memiliki atribut. \textit{Method-method} yang dimiliki kelas ini adalah sebagai berikut.
\begin{itemize}
\item \texttt{public boolean checkPrasyarat(Mahasiswa mahasiswa, java.util.List reasonsContainer)}\\ 


\textbf{Parameter:}\begin{itemize}
\item \texttt{Mahasiswa mahasiswa} - 
\item \texttt{java.util.List reasonsContainer} - 
\end{itemize}
\textbf{Return Value}: Tidak memiliki \textit{return value}

\textbf{Exception}: Tidak memiliki \textit{exception}

\textbf{Override}: \texttt{checkPrasyarat} dari kelas \texttt{MataKuliah}

\end{itemize}
\item \texttt{AIF306}\\ 
Mata kuliah ini bertujuan untuk memberikan pengalaman bagi mahasiswa dalam
 mengerjakan proyek dengan teknologi-teknologi terkini, secara berkelompok.
 Teknologi-teknologi yang digunakan pada kuliah ini tidak spesifik dan dapat
 berubah seiring perkembangan teknologi maupun disesuaikan dengan kompetensi
 dosen pengajar. Beberapa teknologi yang dapat dimanfaatkan antara lain: DVCS
 tool menggunakan Git + Github, Mobile native app (Android, iOS, dll), dan
 responsive web design.

Kelas ini tidak memiliki atribut. \textit{Method-method} yang dimiliki kelas ini adalah sebagai berikut.
\begin{itemize}
\item \texttt{public boolean checkPrasyarat(Mahasiswa mahasiswa, java.util.List reasonsContainer)}\\ 


\textbf{Parameter:}\begin{itemize}
\item \texttt{Mahasiswa mahasiswa} - 
\item \texttt{java.util.List reasonsContainer} - 
\end{itemize}
\textbf{Return Value}: Tidak memiliki \textit{return value}

\textbf{Exception}: Tidak memiliki \textit{exception}

\textbf{Override}: \texttt{checkPrasyarat} dari kelas \texttt{MataKuliah}

\end{itemize}
\item \texttt{AKS122}\\ 


Kelas ini tidak memiliki atribut. Kelas ini tidak memiliki method. \item \texttt{MKU002}\\ 
Pendidikan Kewarganegaraan menjelaskan pentingnya pemahaman tentang identitas nasional 
 Indonesia, hak dan kewajiban warga negara Indonesia serta hubungannya dengan hak dan 
 kewajiban asasi manusia. Materi kuliah mencakup juga wawasan nusantara, ketahanan nasional, 
 politik dan strategi nasional, serta implementasinya dalam kehidupan bermasyarakat, berbangsa 
 dan bernegara kesatuan Republik Indonesia.

Kelas ini tidak memiliki atribut. Kelas ini tidak memiliki method. \item \texttt{AIF462}\\ 


Kelas ini tidak memiliki atribut. Kelas ini tidak memiliki method. \item \texttt{AIF360}\\ 


Kelas ini tidak memiliki atribut. \textit{Method-method} yang dimiliki kelas ini adalah sebagai berikut.
\begin{itemize}
\item \texttt{public boolean checkPrasyarat(Mahasiswa mahasiswa, java.util.List reasonsContainer)}\\ 


\textbf{Parameter:}\begin{itemize}
\item \texttt{Mahasiswa mahasiswa} - 
\item \texttt{java.util.List reasonsContainer} - 
\end{itemize}
\textbf{Return Value}: Tidak memiliki \textit{return value}

\textbf{Exception}: Tidak memiliki \textit{exception}

\textbf{Override}: \texttt{checkPrasyarat} dari kelas \texttt{MataKuliah}

\end{itemize}
\item \texttt{AIF337}\\ 


Kelas ini tidak memiliki atribut. Kelas ini tidak memiliki method. \item \texttt{AIF458}\\ 


Kelas ini tidak memiliki atribut. \textit{Method-method} yang dimiliki kelas ini adalah sebagai berikut.
\begin{itemize}
\item \texttt{public boolean checkPrasyarat(Mahasiswa mahasiswa, java.util.List reasonsContainer)}\\ 


\textbf{Parameter:}\begin{itemize}
\item \texttt{Mahasiswa mahasiswa} - 
\item \texttt{java.util.List reasonsContainer} - 
\end{itemize}
\textbf{Return Value}: Tidak memiliki \textit{return value}

\textbf{Exception}: Tidak memiliki \textit{exception}

\textbf{Override}: \texttt{checkPrasyarat} dari kelas \texttt{MataKuliah}

\end{itemize}
\item \texttt{AIF301}\\ 
Mata kuliah ini memperkenalkan kepada mahasiswa konsep dasar tentang *sistem
 cerdas dan komponen-komponennya. " "Terdapat 4 topik utama yang dibahas yaitu
 teknik pencarian untuk *penyelesaian masalah, representasi pengetahuan dalam
 sistem *cerdas, pemodelan ketidakpastian dalam masalah dan teknik
 pembelajaran mesin.

Kelas ini tidak memiliki atribut. \textit{Method-method} yang dimiliki kelas ini adalah sebagai berikut.
\begin{itemize}
\item \texttt{public boolean checkPrasyarat(Mahasiswa mahasiswa, java.util.List reasonsContainer)}\\ 


\textbf{Parameter:}\begin{itemize}
\item \texttt{Mahasiswa mahasiswa} - 
\item \texttt{java.util.List reasonsContainer} - 
\end{itemize}
\textbf{Return Value}: Tidak memiliki \textit{return value}

\textbf{Exception}: Tidak memiliki \textit{exception}

\textbf{Override}: \texttt{checkPrasyarat} dari kelas \texttt{MataKuliah}

\end{itemize}
\item \texttt{AIF182}\\ 


Kelas ini tidak memiliki atribut. Kelas ini tidak memiliki method. \item \texttt{ESM204}\\ 


Kelas ini tidak memiliki atribut. Kelas ini tidak memiliki method. \item \texttt{AIF205}\\ 
Mata kuliah ini memperkenalkan kepada mahasiswa arsitektur komputer 
 sederhana, modern, dan Advance. Perbedaan, kelebihan dan kekurangan untuk 
 masing-masing arsitektur. Selain itu mahasiswa juga mempelajari cara kerja 
 dari komponen-komponen komputer, terutama memory, cache, system BUS dan 
 input/output.

Kelas ini tidak memiliki atribut. \textit{Method-method} yang dimiliki kelas ini adalah sebagai berikut.
\begin{itemize}
\item \texttt{public boolean checkPrasyarat(Mahasiswa mahasiswa, java.util.List reasonsContainer)}\\ 


\textbf{Parameter:}\begin{itemize}
\item \texttt{Mahasiswa mahasiswa} - 
\item \texttt{java.util.List reasonsContainer} - 
\end{itemize}
\textbf{Return Value}: Tidak memiliki \textit{return value}

\textbf{Exception}: Tidak memiliki \textit{exception}

\textbf{Override}: \texttt{checkPrasyarat} dari kelas \texttt{MataKuliah}

\end{itemize}
\item \texttt{AIF317}\\ 


Kelas ini tidak memiliki atribut. \textit{Method-method} yang dimiliki kelas ini adalah sebagai berikut.
\begin{itemize}
\item \texttt{public boolean checkPrasyarat(Mahasiswa mahasiswa, java.util.List reasonsContainer)}\\ 


\textbf{Parameter:}\begin{itemize}
\item \texttt{Mahasiswa mahasiswa} - 
\item \texttt{java.util.List reasonsContainer} - 
\end{itemize}
\textbf{Return Value}: Tidak memiliki \textit{return value}

\textbf{Exception}: Tidak memiliki \textit{exception}

\textbf{Override}: \texttt{checkPrasyarat} dari kelas \texttt{MataKuliah}

\end{itemize}
\item \texttt{AIF383}\\ 


Kelas ini tidak memiliki atribut. Kelas ini tidak memiliki method. \item \texttt{AIF442}\\ 


Kelas ini tidak memiliki atribut. \textit{Method-method} yang dimiliki kelas ini adalah sebagai berikut.
\begin{itemize}
\item \texttt{public boolean checkPrasyarat(Mahasiswa mahasiswa, java.util.List reasonsContainer)}\\ 


\textbf{Parameter:}\begin{itemize}
\item \texttt{Mahasiswa mahasiswa} - 
\item \texttt{java.util.List reasonsContainer} - 
\end{itemize}
\textbf{Return Value}: Tidak memiliki \textit{return value}

\textbf{Exception}: Tidak memiliki \textit{exception}

\textbf{Override}: \texttt{checkPrasyarat} dari kelas \texttt{MataKuliah}

\end{itemize}
\item \texttt{ESM101}\\ 


Kelas ini tidak memiliki atribut. Kelas ini tidak memiliki method. \item \texttt{AIF403}\\ 
1. Memberikan wawasan kepada mahasiswa tentang kemunculan dan pemanfaatan teknologi baru, 
 khususnya yang berkaitan dengan komputer, dan dampaknya terhadap masyarakat luas.
 2. Memberikan kesadaran dan panduan bersikap kepada mahasiswa dalam menghadapi gejolak yang
 disebabkan oleh munculnya teknologi baru, khususnya yang berkaitan dengan komputer.

Kelas ini tidak memiliki atribut. \textit{Method-method} yang dimiliki kelas ini adalah sebagai berikut.
\begin{itemize}
\item \texttt{public boolean checkPrasyarat(Mahasiswa mahasiswa, java.util.List reasonsContainer)}\\ 


\textbf{Parameter:}\begin{itemize}
\item \texttt{Mahasiswa mahasiswa} - 
\item \texttt{java.util.List reasonsContainer} - 
\end{itemize}
\textbf{Return Value}: Tidak memiliki \textit{return value}

\textbf{Exception}: Tidak memiliki \textit{exception}

\textbf{Override}: \texttt{checkPrasyarat} dari kelas \texttt{MataKuliah}

\end{itemize}
\item \texttt{AIF402}\\ 


Kelas ini tidak memiliki atribut. \textit{Method-method} yang dimiliki kelas ini adalah sebagai berikut.
\begin{itemize}
\item \texttt{public boolean checkPrasyarat(Mahasiswa mahasiswa, java.util.List reasonsContainer)}\\ 


\textbf{Parameter:}\begin{itemize}
\item \texttt{Mahasiswa mahasiswa} - 
\item \texttt{java.util.List reasonsContainer} - 
\end{itemize}
\textbf{Return Value}: Tidak memiliki \textit{return value}

\textbf{Exception}: Tidak memiliki \textit{exception}

\textbf{Override}: \texttt{checkPrasyarat} dari kelas \texttt{MataKuliah}

\end{itemize}
\item \texttt{AIF455}\\ 


Kelas ini tidak memiliki atribut. Kelas ini tidak memiliki method. \item \texttt{AIF443}\\ 


Kelas ini tidak memiliki atribut. Kelas ini tidak memiliki method. \item \texttt{AIF101}\\ 
Mata kuliah ini memperkenalkan kepada mahasiswa konsep dasar pemrograman 
 seperti pengulangan dan percabangan, konsep dasar penyimpanan data kontigu 
 menggunakan array, konsep dasar pemrograman berorientasi objek seperti kelas 
 \& objek, method, dll, termasuk di dalamnya 4 prinsip dasar pemrograman 
 berorientasi objek : data abstraction, encapsulation, inheritance dan 
 polymorphism. Selain, itu diberikan masalah-masalah komputasi sederhana 
 yang harus diselesaikan menggunakan konsep-konsep yang  sudah diperkenalkan 
 dan mengimplementasikannya menggunakan bahasa pemrograman Java

Kelas ini tidak memiliki atribut. Kelas ini tidak memiliki method. \item \texttt{AIF382}\\ 


Kelas ini tidak memiliki atribut. Kelas ini tidak memiliki method. \item \texttt{AIF480}\\ 


Kelas ini tidak memiliki atribut. Kelas ini tidak memiliki method. \item \texttt{AIF316}\\ 
Mata kuliah ini memperkenalkan konsep-konsep dasar komputasi paralel, dimana sebuah 
 program yang berjalan secara paralel harus memiliki safety property dan liveness property. 
 Mahasiswa dikenalkan dengan beberapa teknik pemrograman multi-thread
 seperti lock, monitor, barrier, thread pool, dan sebagainya, yang diimplementasikan 
 dalam bahasa pemrograman Java. Mahasiswa juga dikenalkan dengan beberapa metode untuk 
 menganalisis kebenaran program baik secara matematis maupun secara praktis dengan bantuan 
 model checker.

Kelas ini tidak memiliki atribut. \textit{Method-method} yang dimiliki kelas ini adalah sebagai berikut.
\begin{itemize}
\item \texttt{public boolean checkPrasyarat(Mahasiswa mahasiswa, java.util.List reasonsContainer)}\\ 


\textbf{Parameter:}\begin{itemize}
\item \texttt{Mahasiswa mahasiswa} - 
\item \texttt{java.util.List reasonsContainer} - 
\end{itemize}
\textbf{Return Value}: Tidak memiliki \textit{return value}

\textbf{Exception}: Tidak memiliki \textit{exception}

\textbf{Override}: \texttt{checkPrasyarat} dari kelas \texttt{MataKuliah}

\end{itemize}
\item \texttt{AIF438}\\ 
Mata kuliah ini: Memperkenalkan karakteristik dan teknik visualisasi dari
 berbagai jenis data yang dapat dianalisis dengan teknik-teknik data mining;
 mempelajari teknik-teknik penyiapan data untuk berbagai jenis data dan teknik
 data mining; mempraktekkan teknik-teknik penyiapan data untuk menganalisis
 data nyata/simulasi dengan memanfaatkan perangkat lunak aplikasi.

Kelas ini tidak memiliki atribut. \textit{Method-method} yang dimiliki kelas ini adalah sebagai berikut.
\begin{itemize}
\item \texttt{public boolean checkPrasyarat(Mahasiswa mahasiswa, java.util.List reasonsContainer)}\\ 


\textbf{Parameter:}\begin{itemize}
\item \texttt{Mahasiswa mahasiswa} - 
\item \texttt{java.util.List reasonsContainer} - 
\end{itemize}
\textbf{Return Value}: Tidak memiliki \textit{return value}

\textbf{Exception}: Tidak memiliki \textit{exception}

\textbf{Override}: \texttt{checkPrasyarat} dari kelas \texttt{MataKuliah}

\end{itemize}
\item \texttt{AIF204}\\ 
Mata kuliah ini memperkenalkan konsep dan arsitektur DBMS, mengajarkan 
 aljabar relasional dan SQL serta pemanfaatannya pada pemrograman kueri 
 sederhana s/d relatif kompleks. Selain itu, mata kuliah ini juga mengajarkan 
 dan mempraktekkan perancangan basisdata untuk masalah sederhana 
 (lingkup kecil) termasuk pengembangan program aplikasinya;

Kelas ini tidak memiliki atribut. \textit{Method-method} yang dimiliki kelas ini adalah sebagai berikut.
\begin{itemize}
\item \texttt{public boolean checkPrasyarat(Mahasiswa mahasiswa, java.util.List reasonsContainer)}\\ 


\textbf{Parameter:}\begin{itemize}
\item \texttt{Mahasiswa mahasiswa} - 
\item \texttt{java.util.List reasonsContainer} - 
\end{itemize}
\textbf{Return Value}: Tidak memiliki \textit{return value}

\textbf{Exception}: Tidak memiliki \textit{exception}

\textbf{Override}: \texttt{checkPrasyarat} dari kelas \texttt{MataKuliah}

\end{itemize}
\item \texttt{AIF341}\\ 
Mata kuliah ini memperkenalkan kepada mahasiswa konsep dasar 
 jaringan dan aplikasinya di kehidupan sehari-hari. Mahasiswa 
 dikenalkan dengan teknologi-teknologi terbaru di bidang jaringan, 
 sehingga mahasiswa memiliki pengetahuan yang dapat digunakan 
 dalam kehidupan sehari-hari. Selain itu mahasiswa juga 
 diperkenalkan dengan NetAcad, sebuah layanan dari Cisco yang 
 dapat digunakan untuk memenuhi segala macam kebutuhan terkait 
 dengan Cisco Academy.

Kelas ini tidak memiliki atribut. Kelas ini tidak memiliki method. \item \texttt{AIF183}\\ 


Kelas ini tidak memiliki atribut. Kelas ini tidak memiliki method. \item \texttt{AKS124}\\ 


Kelas ini tidak memiliki atribut. Kelas ini tidak memiliki method. \item \texttt{AIF459}\\ 


Kelas ini tidak memiliki atribut. Kelas ini tidak memiliki method. \item \texttt{AIF336}\\ 
Mata kuliah ini merupakan mata kuliah lanjutan dari mata kuliah Keamanan 
 Informasi, dengan titik berat pada materi kriptografi. Mata kuliah ini 
 memperkenalkan tambahan konsep kriptografi, misalnya tentang otentikasi 
 yaitu otentikasi entitas, manajemen kunci, dan bentuk lain dari metode 
 merahasiakan pesan, yaitu dengan menggunakan secret sharing. Selanjutnya, 
 diperkenalkan juga penggunaan kriptografi pada protokol-protokol yang 
 sebenarnya banyak digunakan sehari-hari, misalnya pada e-cash, auction, 
 dan electronic voting.

Kelas ini tidak memiliki atribut. Kelas ini tidak memiliki method. \item \texttt{AIF463}\\ 


Kelas ini tidak memiliki atribut. Kelas ini tidak memiliki method. \item \texttt{AIF208}\\ 
Mata kuliah ini memperkenalkan kepada mahasiswa tahapan rekayasa perangkat 
 lunak, terutama dengan paradigma berorientasi objek, dilengkapi dengan 
 pengenalan tentang manajemen proyek perangkat lunak.
 Selain, itu diberikan deskripsi proyek berskala kecil yang harus dikerjakan 
 oleh mahasiswa dalam kelompok dengan menerapkan teori yang telah 
 dipelajarinya.

Kelas ini tidak memiliki atribut. \textit{Method-method} yang dimiliki kelas ini adalah sebagai berikut.
\begin{itemize}
\item \texttt{public boolean checkPrasyarat(Mahasiswa mahasiswa, java.util.List reasonsContainer)}\\ 


\textbf{Parameter:}\begin{itemize}
\item \texttt{Mahasiswa mahasiswa} - 
\item \texttt{java.util.List reasonsContainer} - 
\end{itemize}
\textbf{Return Value}: Tidak memiliki \textit{return value}

\textbf{Exception}: Tidak memiliki \textit{exception}

\textbf{Override}: \texttt{checkPrasyarat} dari kelas \texttt{MataKuliah}

\end{itemize}
\item \texttt{MKU003}\\ 
Mata kuliah ini membentuk karakteristik mahasiswa sebagai manusia yang memiliki religiusitas
 melalui pendalaman akan makna agama dan beragama, mendeteksi dinamika Wahyu Tuhan dan iman 
 mereka, memahami relasi dengan Tuhan dan sesama, mengenal makna keselamatan dalam konteks 
 Kerajaan Allah, dan mampu menyatakan ajaran Gereja dalam pelayanan terhadap orang miskin dan
 terlantar.

Kelas ini tidak memiliki atribut. Kelas ini tidak memiliki method. \item \texttt{APS309}\\ 
APS302 atau APS309 ?

Kelas ini tidak memiliki atribut. Kelas ini tidak memiliki method. \item \texttt{AIF335}\\ 


Kelas ini tidak memiliki atribut. Kelas ini tidak memiliki method. \item \texttt{AIF362}\\ 


Kelas ini tidak memiliki atribut. \textit{Method-method} yang dimiliki kelas ini adalah sebagai berikut.
\begin{itemize}
\item \texttt{public boolean checkPrasyarat(Mahasiswa mahasiswa, java.util.List reasonsContainer)}\\ 


\textbf{Parameter:}\begin{itemize}
\item \texttt{Mahasiswa mahasiswa} - 
\item \texttt{java.util.List reasonsContainer} - 
\end{itemize}
\textbf{Return Value}: Tidak memiliki \textit{return value}

\textbf{Exception}: Tidak memiliki \textit{exception}

\textbf{Override}: \texttt{checkPrasyarat} dari kelas \texttt{MataKuliah}

\end{itemize}
\item \texttt{AIF460}\\ 


Kelas ini tidak memiliki atribut. Kelas ini tidak memiliki method. \item \texttt{AIF358}\\ 


Kelas ini tidak memiliki atribut. Kelas ini tidak memiliki method. \item \texttt{AMS100}\\ 
Sistem Bilangan, Fungsi, Limit dan Kekontinuan Fungsi, Turunan, Integral, 
 Penggunaan Integral, Sistem Persamaan Linear, Determinan, Vektor, Nilai dan 
 Vektor Eigen.

Kelas ini tidak memiliki atribut. Kelas ini tidak memiliki method. \item \texttt{AIF401}\\ 


Kelas ini tidak memiliki atribut. \textit{Method-method} yang dimiliki kelas ini adalah sebagai berikut.
\begin{itemize}
\item \texttt{public boolean checkPrasyarat(Mahasiswa mahasiswa, java.util.List reasonsContainer)}\\ 


\textbf{Parameter:}\begin{itemize}
\item \texttt{Mahasiswa mahasiswa} - 
\item \texttt{java.util.List reasonsContainer} - 
\end{itemize}
\textbf{Return Value}: Tidak memiliki \textit{return value}

\textbf{Exception}: Tidak memiliki \textit{exception}

\textbf{Override}: \texttt{checkPrasyarat} dari kelas \texttt{MataKuliah}

\end{itemize}
\item \texttt{AIF456}\\ 


Kelas ini tidak memiliki atribut. Kelas ini tidak memiliki method. \item \texttt{SIR104}\\ 


Kelas ini tidak memiliki atribut. Kelas ini tidak memiliki method. \item \texttt{AIF339}\\ 


Kelas ini tidak memiliki atribut. \textit{Method-method} yang dimiliki kelas ini adalah sebagai berikut.
\begin{itemize}
\item \texttt{public boolean checkPrasyarat(Mahasiswa mahasiswa, java.util.List reasonsContainer)}\\ 


\textbf{Parameter:}\begin{itemize}
\item \texttt{Mahasiswa mahasiswa} - 
\item \texttt{java.util.List reasonsContainer} - 
\end{itemize}
\textbf{Return Value}: Tidak memiliki \textit{return value}

\textbf{Exception}: Tidak memiliki \textit{exception}

\textbf{Override}: \texttt{checkPrasyarat} dari kelas \texttt{MataKuliah}

\end{itemize}
\item \texttt{AIF102}\\ 
Mata kuliah ini memperkenalkan berbagai algoritma dan teknik-teknik 
 penyelesaian masalah komputasi seperti rekursif, sorting, teknik divide dan 
 conquer, serta exhaustive search. Selain itu, pada kuliah ini juga 
 dikenalkan berbagai struktur data yang dapat digunakan untuk mendukung 
 penyelesaian masalah komputasi seperti ADT List, Stack dan Queue. Baik 
 algoritma maupun struktur data yang dikenalkan harus dapat diimplementasikan 
 dan digunakan oleh mahasiswa untuk menyelesaikan masalah dengan menggunakan 
 suatu bahasa pemrograman berorientasi objek.

Kelas ini tidak memiliki atribut. \textit{Method-method} yang dimiliki kelas ini adalah sebagai berikut.
\begin{itemize}
\item \texttt{public boolean checkPrasyarat(Mahasiswa mahasiswa, java.util.List reasonsContainer)}\\ 


\textbf{Parameter:}\begin{itemize}
\item \texttt{Mahasiswa mahasiswa} - 
\item \texttt{java.util.List reasonsContainer} - 
\end{itemize}
\textbf{Return Value}: Tidak memiliki \textit{return value}

\textbf{Exception}: Tidak memiliki \textit{exception}

\textbf{Override}: \texttt{checkPrasyarat} dari kelas \texttt{MataKuliah}

\end{itemize}
\item \texttt{AIF381}\\ 


Kelas ini tidak memiliki atribut. Kelas ini tidak memiliki method. \item \texttt{AIF483}\\ 


Kelas ini tidak memiliki atribut. Kelas ini tidak memiliki method. \item \texttt{AIF315}\\ 
Mata kuliah ini memperkenalkan konsep dan lingkungan pemrograman berbasis web,
 kemudian belajar membuat aplikasi berbasis web menggunakan HTML5, CSS, Java Script 
 dan PHP. Untuk meningkatkan keterampilan pemrograman dilengkapi dengan praktikum. 
 Sedangkan untuk mendapatkan pengalaman penerapan konsep diberikan tugas besar membuat 
 program berbasis web dengan kasus yang ditentukan oleh mahasiswa.

Kelas ini tidak memiliki atribut. \textit{Method-method} yang dimiliki kelas ini adalah sebagai berikut.
\begin{itemize}
\item \texttt{public boolean checkPrasyarat(Mahasiswa mahasiswa, java.util.List reasonsContainer)}\\ 


\textbf{Parameter:}\begin{itemize}
\item \texttt{Mahasiswa mahasiswa} - 
\item \texttt{java.util.List reasonsContainer} - 
\end{itemize}
\textbf{Return Value}: Tidak memiliki \textit{return value}

\textbf{Exception}: Tidak memiliki \textit{exception}

\textbf{Override}: \texttt{checkPrasyarat} dari kelas \texttt{MataKuliah}

\end{itemize}
\item \texttt{AIF342}\\ 


Kelas ini tidak memiliki atribut. \textit{Method-method} yang dimiliki kelas ini adalah sebagai berikut.
\begin{itemize}
\item \texttt{public boolean checkPrasyarat(Mahasiswa mahasiswa, java.util.List reasonsContainer)}\\ 


\textbf{Parameter:}\begin{itemize}
\item \texttt{Mahasiswa mahasiswa} - 
\item \texttt{java.util.List reasonsContainer} - 
\end{itemize}
\textbf{Return Value}: Tidak memiliki \textit{return value}

\textbf{Exception}: Tidak memiliki \textit{exception}

\textbf{Override}: \texttt{checkPrasyarat} dari kelas \texttt{MataKuliah}

\end{itemize}
\item \texttt{IIE214}\\ 


Kelas ini tidak memiliki atribut. Kelas ini tidak memiliki method. \item \texttt{AIF303}\\ 
Mempelajari Konsep Data, Informasi, Pengetahuan, Sistem Informasi, proses dan
 pemodelan bisnis, jenis-jenis sistem informasi, untuk mendukung pengambilan
 keputusan. Mempelajari trend Teknologi Informasi, tahap-tahap pembangunan
 sistem informasi. Mempelajari pengantar : EIS, e-bisnis/e-commerce, Business
 Intelligence, Cloud Computing dan Mobile Applications

Kelas ini tidak memiliki atribut. \textit{Method-method} yang dimiliki kelas ini adalah sebagai berikut.
\begin{itemize}
\item \texttt{public boolean checkPrasyarat(Mahasiswa mahasiswa, java.util.List reasonsContainer)}\\ 


\textbf{Parameter:}\begin{itemize}
\item \texttt{Mahasiswa mahasiswa} - 
\item \texttt{java.util.List reasonsContainer} - 
\end{itemize}
\textbf{Return Value}: Tidak memiliki \textit{return value}

\textbf{Exception}: Tidak memiliki \textit{exception}

\textbf{Override}: \texttt{checkPrasyarat} dari kelas \texttt{MataKuliah}

\end{itemize}
\item \texttt{AIF302}\\ 
Mata kuliah ini melatih mahasiswa dalam menulis ilmiah serta memperkenalkan
 metodologi penelitian serta kakas untuk menulis ilmiah.

Kelas ini tidak memiliki atribut. \textit{Method-method} yang dimiliki kelas ini adalah sebagai berikut.
\begin{itemize}
\item \texttt{public boolean checkPrasyarat(Mahasiswa mahasiswa, java.util.List reasonsContainer)}\\ 


\textbf{Parameter:}\begin{itemize}
\item \texttt{Mahasiswa mahasiswa} - 
\item \texttt{java.util.List reasonsContainer} - 
\end{itemize}
\textbf{Return Value}: Tidak memiliki \textit{return value}

\textbf{Exception}: Tidak memiliki \textit{exception}

\textbf{Override}: \texttt{checkPrasyarat} dari kelas \texttt{MataKuliah}

\end{itemize}
\item \texttt{AIF181}\\ 


Kelas ini tidak memiliki atribut. Kelas ini tidak memiliki method. \item \texttt{AIF210}\\ 


Kelas ini tidak memiliki atribut. Kelas ini tidak memiliki method. \item \texttt{AIF343}\\ 


Kelas ini tidak memiliki atribut. Kelas ini tidak memiliki method. \item \texttt{AIF206}\\ 
Mata kuliah ini memperkenalkan kepada mahasiswa mengenai konsep sistem 
 operasi, jenis-jenis sistem operasi yang digunakan dalam kehidupan 
 sehari-hari dan beberapa perangkat keras yang dibutuhkan pada komputer. 
 Selain itu juga mempelajari mengenai teknik dan algoritma yang digunakan 
 dalam pengelolaan sistem operasi.

Kelas ini tidak memiliki atribut. \textit{Method-method} yang dimiliki kelas ini adalah sebagai berikut.
\begin{itemize}
\item \texttt{public boolean checkPrasyarat(Mahasiswa mahasiswa, java.util.List reasonsContainer)}\\ 


\textbf{Parameter:}\begin{itemize}
\item \texttt{Mahasiswa mahasiswa} - 
\item \texttt{java.util.List reasonsContainer} - 
\end{itemize}
\textbf{Return Value}: Tidak memiliki \textit{return value}

\textbf{Exception}: Tidak memiliki \textit{exception}

\textbf{Override}: \texttt{checkPrasyarat} dari kelas \texttt{MataKuliah}

\end{itemize}
\item \texttt{AIF314}\\ 
Kuliah ini merupakan kelanjutan dari kuliah Manajemen Informasi Basisdata.
 Pada perkuliahan ini, mahasiswa akan mempelajari teknik-teknik pengelolaan
 basis data dan membuat program dengan basis data yang optimal/efisien.

Kelas ini tidak memiliki atribut. \textit{Method-method} yang dimiliki kelas ini adalah sebagai berikut.
\begin{itemize}
\item \texttt{public boolean checkPrasyarat(Mahasiswa mahasiswa, java.util.List reasonsContainer)}\\ 


\textbf{Parameter:}\begin{itemize}
\item \texttt{Mahasiswa mahasiswa} - 
\item \texttt{java.util.List reasonsContainer} - 
\end{itemize}
\textbf{Return Value}: Tidak memiliki \textit{return value}

\textbf{Exception}: Tidak memiliki \textit{exception}

\textbf{Override}: \texttt{checkPrasyarat} dari kelas \texttt{MataKuliah}

\end{itemize}
\item \texttt{AIF380}\\ 


Kelas ini tidak memiliki atribut. Kelas ini tidak memiliki method. \item \texttt{AIF103}\\ 
Mata kuliah ini merupakan salah satu cara untuk mencapai kompetensi dasar 
 tentang matematika diskrit yang prinsipnya banyak digunakan dalam bidang 
 ilmu komputer. Selain itu, kuliah ini juga merupakan cara untuk membentuk 
 pola pikir logis yang dibutuhkan untuk menempuh kuliah-kuliah di tingkat 
 yang lebih tinggi.

Kelas ini tidak memiliki atribut. Kelas ini tidak memiliki method. \item \texttt{AIF292}\\ 


Kelas ini tidak memiliki atribut. Kelas ini tidak memiliki method. \item \texttt{AIF441}\\ 
Mata kuliah ini memperkenalkan kepada mahasiswa konsep jaringan lanjut
 terutama di layer data link dan layer network. Materi utama dari mata kuliah
 ini adalah pengembangan jaringan dan pengenalan fungsi-fungsi yang terdapat
 pada alat jaringan Cisco yang berkaitan dengan layer 2 dan layer 3.

Kelas ini tidak memiliki atribut. \textit{Method-method} yang dimiliki kelas ini adalah sebagai berikut.
\begin{itemize}
\item \texttt{public boolean checkPrasyarat(Mahasiswa mahasiswa, java.util.List reasonsContainer)}\\ 


\textbf{Parameter:}\begin{itemize}
\item \texttt{Mahasiswa mahasiswa} - 
\item \texttt{java.util.List reasonsContainer} - 
\end{itemize}
\textbf{Return Value}: Tidak memiliki \textit{return value}

\textbf{Exception}: Tidak memiliki \textit{exception}

\textbf{Override}: \texttt{checkPrasyarat} dari kelas \texttt{MataKuliah}

\end{itemize}
\item \texttt{AIF457}\\ 
Mata kuliah ini memperkenalkan konsep kewirausahaan dengan memanfaatkan teknologi, khususnya
 teknologi informasi, sebagai basis usaha dan inovasi produk/jasa; Mempelajari
 teknik mencari peluang dan merumuskan bidang usaha spesifik yang akan
 diterjuni; Mempelajari konsep manajemen pemasaran, keuangan dan SDM dalam
 kaitannya dengan berwira-usaha di bidang TI; Menyusun proposal bisnis untuk
 berwira-usaha di bidang TI dan mempresentasikannya.

Kelas ini tidak memiliki atribut. \textit{Method-method} yang dimiliki kelas ini adalah sebagai berikut.
\begin{itemize}
\item \texttt{public boolean checkPrasyarat(Mahasiswa mahasiswa, java.util.List reasonsContainer)}\\ 


\textbf{Parameter:}\begin{itemize}
\item \texttt{Mahasiswa mahasiswa} - 
\item \texttt{java.util.List reasonsContainer} - 
\end{itemize}
\textbf{Return Value}: Tidak memiliki \textit{return value}

\textbf{Exception}: Tidak memiliki \textit{exception}

\textbf{Override}: \texttt{checkPrasyarat} dari kelas \texttt{MataKuliah}

\end{itemize}
\item \texttt{MKU001}\\ 
Mata Kuliah Pendidikan Pancasila berupaya menelaah/mengkaji berbagai fenomena kehidupan 
 bangsa dan Negara Indonesia sebagai sebuah ruang publik dengan menggunakan pendekatan 
 hermeneutika (filsafat) dan pendidikan nilai (pedagogik). Dengan bantuan hermenutika
 mahasiswa diajak berpikir kritis terhadap segala bentuk ideologisme Pancasila dan melalui 
 pendidikan nilai mahasiswa dilatih untuk memiliki nilai Pancasila. Nilai pengembangan diri 
 intra-personal dan relasi inter-personal dapat tertanam melalui pendidikan Pancasila yang 
 tujuannya adalah membangun kepribadian (character building) manusia Indonesia yang utuh, 
 baik menyangkut aspek kognitif, afektif, maupun psikomotor. Dengan demikian, Pendidikan 
 Pancasila mengajak mahasiswa menilai realitas ruang publik sehari-hari secara mandiri 
 dengan panduan nilai-nilai etis Pancasila.

Kelas ini tidak memiliki atribut. Kelas ini tidak memiliki method. \item \texttt{ESA101}\\ 


Kelas ini tidak memiliki atribut. Kelas ini tidak memiliki method. \item \texttt{AMS290}\\ 


Kelas ini tidak memiliki atribut. Kelas ini tidak memiliki method. \item \texttt{AIF461}\\ 


Kelas ini tidak memiliki atribut. Kelas ini tidak memiliki method. \item \texttt{AIF318}\\ 
Mata kuliah ini memperkenalkan konsep perangkat mobile dan pemrograman pada perangkat 
 mobile. Pemrograman dikhususkan pada lingkungan J2ME dan Android.
 Untuk meningkatkan keterampilan pemrograman dilengkapi dengan praktikum. 
 Sedangkan untuk mendapatkan pengalaman penerapan konsep diberikan tugas implementasi suatu 
 kasus pada lingkungan mobile-cloud dengan kasus yang sudah ditentukan.

Kelas ini tidak memiliki atribut. \textit{Method-method} yang dimiliki kelas ini adalah sebagai berikut.
\begin{itemize}
\item \texttt{public boolean checkPrasyarat(Mahasiswa mahasiswa, java.util.List reasonsContainer)}\\ 


\textbf{Parameter:}\begin{itemize}
\item \texttt{Mahasiswa mahasiswa} - 
\item \texttt{java.util.List reasonsContainer} - 
\end{itemize}
\textbf{Return Value}: Tidak memiliki \textit{return value}

\textbf{Exception}: Tidak memiliki \textit{exception}

\textbf{Override}: \texttt{checkPrasyarat} dari kelas \texttt{MataKuliah}

\end{itemize}
\item \texttt{AIF334}\\ 


Kelas ini tidak memiliki atribut. Kelas ini tidak memiliki method. \item \texttt{ESM105}\\ 


Kelas ini tidak memiliki atribut. Kelas ini tidak memiliki method. \item \texttt{AIF450}\\ 


Kelas ini tidak memiliki atribut. Kelas ini tidak memiliki method. \item \texttt{AIF446}\\ 


Kelas ini tidak memiliki atribut. Kelas ini tidak memiliki method. \item \texttt{AIF104}\\ 
Mata kuliah ini memberikan pengetahuan tentang logika yang digunakan di 
 dalam ilmu komputer. Dalam kuliah ini, mahasiswa belajar untuk bisa 
 memodelkan suatu kalimat dalam kehidupan sehari-hari, ke dalam kalimat 
 dengan sintaks tertentu, yang hanya memiliki satu arti. Lalu, diperkenalkan 
 juga, bagaimana mengartikan suatu kalimat (benar atau salah) dan bagaimana 
 menentukan sifat dari kalimat tersebut.

Kelas ini tidak memiliki atribut. Kelas ini tidak memiliki method. \item \texttt{AIF387}\\ 


Kelas ini tidak memiliki atribut. Kelas ini tidak memiliki method. \item \texttt{AIF313}\\ 
Mata kuliah ini memperkenalkan kepada mahasiswa konsep dasar pembuatan grafik
 dengan menggunakan komputer seperti mengenal berbagai algoritma pembuatan
 primitif 2 dimensi seperti titik, garis, lingkaran, elips, berbagai macam
 bentuk kurva, fraktal, konsep warna (RGB), dasar-dasar grafika 3 dimensi
 seperti pewarnaan, pencahayaan, pemberian tekstur pada objek, transformasi,
 animasi, dan sebagainya. Selain, itu diberikan masalah-masalah komputasi
 sederhana yang harus diselesaikan menggunakan konsep-konsep yang sudah
 diperkenalkan dan mengimplementasikannya menggunakan bahasa pemrograman Java.

Kelas ini tidak memiliki atribut. Kelas ini tidak memiliki method. \item \texttt{AIF344}\\ 


Kelas ini tidak memiliki atribut. \textit{Method-method} yang dimiliki kelas ini adalah sebagai berikut.
\begin{itemize}
\item \texttt{public boolean checkPrasyarat(Mahasiswa mahasiswa, java.util.List reasonsContainer)}\\ 


\textbf{Parameter:}\begin{itemize}
\item \texttt{Mahasiswa mahasiswa} - 
\item \texttt{java.util.List reasonsContainer} - 
\end{itemize}
\textbf{Return Value}: Tidak memiliki \textit{return value}

\textbf{Exception}: Tidak memiliki \textit{exception}

\textbf{Override}: \texttt{checkPrasyarat} dari kelas \texttt{MataKuliah}

\end{itemize}
\item \texttt{AIF201}\\ 
Mata kuliah ini memperkenalkan prinsip-prinsip yang digunakan dalam 
 melakukan analisa serta desain prorgram berorientasi objek. Di samping itu, 
 mahasiswa juga belajar menggunakan kakas berupa diagram UML (Unified 
 Modelling Language) sehingga dapat mengkomunikasikan desain secara visual. 
 Mahasiswa juga akan mengenal beberapa software design pattern dari Gang of 
 Four. Terakhir, mahasiswa akan belajar mengenai konsep MVC 
 (Model-View-Controller) yang menjadi dasar dari banyak framework masa kini.
 Bahasa yang digunakan adalah bahasa Java, namun diusahakan tetap umum 
 sehingga dapat diaplikasikan pada bahasa yang lain.

Kelas ini tidak memiliki atribut. \textit{Method-method} yang dimiliki kelas ini adalah sebagai berikut.
\begin{itemize}
\item \texttt{public boolean checkPrasyarat(Mahasiswa mahasiswa, java.util.List reasonsContainer)}\\ 


\textbf{Parameter:}\begin{itemize}
\item \texttt{Mahasiswa mahasiswa} - 
\item \texttt{java.util.List reasonsContainer} - 
\end{itemize}
\textbf{Return Value}: Tidak memiliki \textit{return value}

\textbf{Exception}: Tidak memiliki \textit{exception}

\textbf{Override}: \texttt{checkPrasyarat} dari kelas \texttt{MataKuliah}

\end{itemize}
\item \texttt{AIF352}\\ 


Kelas ini tidak memiliki atribut. Kelas ini tidak memiliki method. \item \texttt{AIF305}\\ 
Mata kuliah ini memperkenalkan kepada mahasiswa konsep dasar jaringan
 komputer dengan menggunakan top-down. Selain itu mengajarkan juga kepada
 mahasiswa mengenai aplikasi-aplikasi berbasis jaringan sehingga diharapkan
 mahasiswa dapat membuat aplikasi berbasis jaringan dengan menggunakan socket.
 Pada akhirnya, mahasiswa akan ditugaskan untuk membangun jaringan komputer
 LAN, baik menggunakan kabel maupun nirkabel.

Kelas ini tidak memiliki atribut. \textit{Method-method} yang dimiliki kelas ini adalah sebagai berikut.
\begin{itemize}
\item \texttt{public boolean checkPrasyarat(Mahasiswa mahasiswa, java.util.List reasonsContainer)}\\ 


\textbf{Parameter:}\begin{itemize}
\item \texttt{Mahasiswa mahasiswa} - 
\item \texttt{java.util.List reasonsContainer} - 
\end{itemize}
\textbf{Return Value}: Tidak memiliki \textit{return value}

\textbf{Exception}: Tidak memiliki \textit{exception}

\textbf{Override}: \texttt{checkPrasyarat} dari kelas \texttt{MataKuliah}

\end{itemize}
\item \texttt{MKU010}\\ 
Mata kuliah ini difokuskan pada pemahaman sumber referensi dalam Bahasa Inggris dan 
 pengembangan kosakata Bahasa Inggris (vocabularies). Hampir keseluruhan waktu perkuliahan 
 didedikasikan untuk menjelaskan metode mengekstraksi isi bacaan secara tepat dan melatih 
 mahasiswa untuk menerapkan metode tersebut seraya menambah kosakata-kosakata baru. 
 Mahasiswa juga dilatih untuk mempresentasikan hasil pemahamannya akan isi bahan bacaan.

Kelas ini tidak memiliki atribut. Kelas ini tidak memiliki method. \item \texttt{MKU011}\\ 
Mata kuliah estetika memberi pemahaman konseptual filosofis "seni" dalam khasanah keilmuan, 
 pembentukan kesadaran ekologis juga dalam proses pembudayaan dan peradaban. Mata kuliah ini 
 akan menjadi fondasi bagi mahasiswa untuk memahami dan mempraktekkan seni dari sudut pandang
 filsafat, sejarah, kultural, dan global. Melalui mata kuliah ini, mahasiswa mempelajari
 mengenai dunia manusia (manusia dan pikirannya), pluralitas dan relativitas seni, serta 
 aliran-aliran seni rupa Barat.

Kelas ini tidak memiliki atribut. Kelas ini tidak memiliki method. \item \texttt{AIF332}\\ 


Kelas ini tidak memiliki atribut. \textit{Method-method} yang dimiliki kelas ini adalah sebagai berikut.
\begin{itemize}
\item \texttt{public boolean checkPrasyarat(Mahasiswa mahasiswa, java.util.List reasonsContainer)}\\ 


\textbf{Parameter:}\begin{itemize}
\item \texttt{Mahasiswa mahasiswa} - 
\item \texttt{java.util.List reasonsContainer} - 
\end{itemize}
\textbf{Return Value}: Tidak memiliki \textit{return value}

\textbf{Exception}: Tidak memiliki \textit{exception}

\textbf{Override}: \texttt{checkPrasyarat} dari kelas \texttt{MataKuliah}

\end{itemize}
\item \texttt{AIF304}\\ 
Mata kuliah ini memberikan kesempatan bagi mahasiswa untuk memperdalam konsep
 tentang pengembangan sistem informasi dan mempraktekkan analisis kebutuhan,
 analisis sistem dan perancangan sitem pada organisasi studi kasus;

Kelas ini tidak memiliki atribut. \textit{Method-method} yang dimiliki kelas ini adalah sebagai berikut.
\begin{itemize}
\item \texttt{public boolean checkPrasyarat(Mahasiswa mahasiswa, java.util.List reasonsContainer)}\\ 


\textbf{Parameter:}\begin{itemize}
\item \texttt{Mahasiswa mahasiswa} - 
\item \texttt{java.util.List reasonsContainer} - 
\end{itemize}
\textbf{Return Value}: Tidak memiliki \textit{return value}

\textbf{Exception}: Tidak memiliki \textit{exception}

\textbf{Override}: \texttt{checkPrasyarat} dari kelas \texttt{MataKuliah}

\end{itemize}
\item \texttt{ESM201}\\ 


Kelas ini tidak memiliki atribut. Kelas ini tidak memiliki method. \item \texttt{AIF312}\\ 
Mata kuliah ini memberikan pengetahuan awal tentang keamanan informasi. Pada
 beberapa pertemuan awal, dibahas keamanan informasi secara matematis, yaitu
 di materi-materi seputar kriptografi dan serangannya. Lalu, dibahas pula
 konsep keamanan informasi pada jaringan komputer dan pada software.

Kelas ini tidak memiliki atribut. \textit{Method-method} yang dimiliki kelas ini adalah sebagai berikut.
\begin{itemize}
\item \texttt{public boolean checkPrasyarat(Mahasiswa mahasiswa, java.util.List reasonsContainer)}\\ 


\textbf{Parameter:}\begin{itemize}
\item \texttt{Mahasiswa mahasiswa} - 
\item \texttt{java.util.List reasonsContainer} - 
\end{itemize}
\textbf{Return Value}: Tidak memiliki \textit{return value}

\textbf{Exception}: Tidak memiliki \textit{exception}

\textbf{Override}: \texttt{checkPrasyarat} dari kelas \texttt{MataKuliah}

\end{itemize}
\item \texttt{AIF484}\\ 


Kelas ini tidak memiliki atribut. Kelas ini tidak memiliki method. \item \texttt{AIF191}\\ 


Kelas ini tidak memiliki atribut. Kelas ini tidak memiliki method. \item \texttt{AIF386}\\ 


Kelas ini tidak memiliki atribut. Kelas ini tidak memiliki method. \item \texttt{AIF105}\\ 
Mata kuliah ini memperkenalkan kepada mahasiswa terminologi dan konsep dasar 
 yang akan banyak dipakai sepanjang kuliah di Teknik Informatika. Selain itu 
 mata kuliah ini juga mempersiapkan dan membiasakan mahasiswa dengan suasana 
 akademik yang khas perguruan tinggi seperti kedisiplinan, kerja sama, 
 kemampuan menggunakan teknologi informasi dalam pembuatan tugas, kemampuan 
 komunikasi, dsb.

Kelas ini tidak memiliki atribut. Kelas ini tidak memiliki method. \item \texttt{AIF294}\\ 


Kelas ini tidak memiliki atribut. Kelas ini tidak memiliki method. \item \texttt{AIF282}\\ 


Kelas ini tidak memiliki atribut. Kelas ini tidak memiliki method. \item \texttt{AIF451}\\ 


Kelas ini tidak memiliki atribut. Kelas ini tidak memiliki method. \item \texttt{APS302}\\ 


Kelas ini tidak memiliki atribut. Kelas ini tidak memiliki method. \item \texttt{EAA101}\\ 


Kelas ini tidak memiliki atribut. Kelas ini tidak memiliki method. \item \texttt{HasPrasyarat}\\ 
Mendefinisikan kelas-kelas yang memiliki prasyarat, terkustomisasi
 untuk seorang \texttt{Mahasiswa}. Jika ada tambahan, jangan lupa untuk
 mendaftarkannya di DEFAULT\_HASPRASYARAT\_CLASSES. Jika berubah package,
 jangan lupa mengupdate DEFAULT\_PRASYARAT\_PACKAGE.

Atribut yang dimiliki kelas ini adalah sebagai berikut.
\begin{itemize}
\item \texttt{String DEFAULT\_HASPRASYARAT\_CLASSES} - Daftar dari nama kelas default seluruh turunan interface ini. Perlu didaftarkan
 manual, karena Java reflection tidak dapat mendeteksi otomatis.
\item \texttt{String DEFAULT\_PRASYARAT\_PACKAGE} - Package tempat menyimpan seluruh turunan standar interface ini. Perlu didefinisikan
 manual, karena Java reflection tidak dapat mendeteksi otomatis.
\end{itemize}
\textit{Method-method} yang dimiliki kelas ini adalah sebagai berikut.
\begin{itemize}
\item \texttt{public boolean checkPrasyarat(Mahasiswa mahasiswa, java.util.List reasonsContainer)}\\ 
Memeriksa prasyarat-prasyarat dari kuliah, spesifik untuk mahasiswa
 yang dituju. Jika ada pesan-pesan khusus, akan ditambahkan pada parameter
 reasonsContainer.

\textbf{Parameter:}
\begin{itemize}
\item \texttt{Mahasiswa mahasiswa} - 
prasyarat kuliah akan diperiksa spesifik pada mahasiswa ini
\item \texttt{java.util.List reasonsContainer} - 
pesan-pesan terkait prasyarat akan ditambahkan di sini, jika ada.
\end{itemize}
\textbf{Return Value}: true jika seluruh prasyarat dipenuhi, false jika tidak.

\textbf{Exception}: Tidak memiliki \textit{exception}

\end{itemize}
\item \texttt{HasPraktikum}\\ 


Kelas ini tidak memiliki atribut. Kelas ini tidak memiliki method. \item \texttt{HasResponsi}\\ 


Kelas ini tidak memiliki atribut. Kelas ini tidak memiliki method. \item \texttt{Kelulusan}\\ 


Atribut yang dimiliki kelas ini adalah sebagai berikut.
\begin{itemize}
\item \texttt{String PILIHAN\_WAJIB} - 
\item \texttt{String WAJIB} - 
\item \texttt{String AGAMA} - 
\item \texttt{int MIN\_SKS\_LULUS} - 
\item \texttt{int MIN\_PILIHAN\_WAJIB} - 
\end{itemize}
\textit{Method-method} yang dimiliki kelas ini adalah sebagai berikut.
\begin{itemize}
\item \texttt{public boolean checkPrasyarat(Mahasiswa mahasiswa, java.util.List reasonsContainer)}\\ 


\textbf{Parameter:}\begin{itemize}
\item \texttt{Mahasiswa mahasiswa} - 
\item \texttt{java.util.List reasonsContainer} - 
\end{itemize}
\textbf{Return Value}: Tidak memiliki \textit{return value}

\textbf{Exception}: Tidak memiliki \textit{exception}

\textbf{Override}: \texttt{checkPrasyarat} dari kelas \texttt{Object}

\end{itemize}
\end{enumerate}
\end{document}
